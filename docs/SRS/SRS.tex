% THIS DOCUMENT IS TAILORED TO REQUIREMENTS FOR SCIENTIFIC COMPUTING.  IT SHOULDN'T
% BE USED FOR NON-SCIENTIFIC COMPUTING PROJECTS
\documentclass[12pt]{article}

\usepackage{amsmath, mathtools}
\usepackage{amsfonts}
\usepackage{amssymb}
\usepackage{graphicx}
\usepackage{colortbl}
\usepackage{xr}
\usepackage{hyperref}
\usepackage{longtable}
\usepackage{xfrac}
\usepackage{tabularx}
\usepackage{float}
\usepackage{siunitx}
\usepackage{booktabs}
\usepackage{caption}
\usepackage{pdflscape}
\usepackage{afterpage}

\usepackage[round]{natbib}

%\usepackage{refcheck}

\hypersetup{
    bookmarks=true,         % show bookmarks bar?
      colorlinks=true,       % false: boxed links; true: colored links
    linkcolor=red,          % color of internal links (change box color with linkbordercolor)
    citecolor=green,        % color of links to bibliography
    filecolor=magenta,      % color of file links
    urlcolor=cyan           % color of external links
}

\input{../Comments}
%% Common Parts

\newcommand{\progname}{SFWRENG 4G06} % PUT YOUR PROGRAM NAME HERE
\newcommand{\authname}{Team \#7, Team FAAM, SweatSmart
\\
\\ Daniel Akselrod
\\ Jonathan Avraham
\\ Sophie Fillion
\\ Sam McDonald} % AUTHOR NAMES                  

\usepackage{hyperref}
\hypersetup{colorlinks=true, linkcolor=blue, citecolor=blue, filecolor=blue,
    urlcolor=blue, unicode=false}
\urlstyle{same}
                                


% For easy change of table widths
\newcommand{\colZwidth}{1.0\textwidth}
\newcommand{\colAwidth}{0.13\textwidth}
\newcommand{\colBwidth}{0.82\textwidth}
\newcommand{\colCwidth}{0.1\textwidth}
\newcommand{\colDwidth}{0.05\textwidth}
\newcommand{\colEwidth}{0.8\textwidth}
\newcommand{\colFwidth}{0.17\textwidth}
\newcommand{\colGwidth}{0.5\textwidth}
\newcommand{\colHwidth}{0.28\textwidth}

% Used so that cross-references have a meaningful prefix
\newcounter{defnum} %Definition Number
\newcommand{\dthedefnum}{GD\thedefnum}
\newcommand{\dref}[1]{GD\ref{#1}}
\newcounter{datadefnum} %Datadefinition Number
\newcommand{\ddthedatadefnum}{DD\thedatadefnum}
\newcommand{\ddref}[1]{DD\ref{#1}}
\newcounter{theorynum} %Theory Number
\newcommand{\tthetheorynum}{TM\thetheorynum}
\newcommand{\tref}[1]{TM\ref{#1}}
\newcounter{tablenum} %Table Number
\newcommand{\tbthetablenum}{TB\thetablenum}
\newcommand{\tbref}[1]{TB\ref{#1}}
\newcounter{assumpnum} %Assumption Number
\newcommand{\atheassumpnum}{A\theassumpnum}
\newcommand{\aref}[1]{A\ref{#1}}
\newcounter{goalnum} %Goal Number
\newcommand{\gthegoalnum}{GS\thegoalnum}
\newcommand{\gsref}[1]{GS\ref{#1}}
\newcounter{instnum} %Instance Number
\newcommand{\itheinstnum}{IM\theinstnum}
\newcommand{\iref}[1]{IM\ref{#1}}
\newcounter{reqnum} %Requirement Number
\newcommand{\rthereqnum}{R\thereqnum}
\newcommand{\rref}[1]{R\ref{#1}}
\newcounter{nfrnum} %NFR Number
\newcommand{\rthenfrnum}{NFR\thenfrnum}
\newcommand{\nfrref}[1]{NFR\ref{#1}}
\newcounter{lcnum} %Likely change number
\newcommand{\lthelcnum}{LC\thelcnum}
\newcommand{\lcref}[1]{LC\ref{#1}}

\usepackage{fullpage}

\newcommand{\deftheory}[9][Not Applicable]
{
\newpage
\noindent \rule{\textwidth}{0.5mm}

\paragraph{RefName: } \textbf{#2} \phantomsection 
\label{#2}

\paragraph{Label:} #3

\noindent \rule{\textwidth}{0.5mm}

\paragraph{Equation:}

#4

\paragraph{Description:}

#5

\paragraph{Notes:}

#6

\paragraph{Source:}

#7

\paragraph{Ref.\ By:}

#8

\paragraph{Preconditions for \hyperref[#2]{#2}:}
\label{#2_precond}

#9

\paragraph{Derivation for \hyperref[#2]{#2}:}
\label{#2_deriv}

#1

\noindent \rule{\textwidth}{0.5mm}

}

\begin{document}

\title{Software Requirements Specification\\ \progname} 
\author{\authname}
\date{\today}
	
\maketitle

~\newpage

\pagenumbering{roman}

\tableofcontents

~\newpage

\section*{Revision History}

\begin{table}[hp]
		\centering
		\begin{tabularx}{\textwidth}{lllX}
			\toprule
			\textbf{Revision Version} & \textbf{Date} & \textbf{Developer(s)} & \textbf{Change}\\
			\midrule
			0 & Oct 6, 2023 & Sophie, Daniel, Sam, Jonathan & First draft\\
			\bottomrule
		\end{tabularx}
	\end{table}


~\newpage

\section{Reference Material}

\subsection{Terminology}
\begin{enumerate}
    \item ML - machine learning
    \item AI - Artificial Intelligence 
    \item Rep - short for repetition 
\end{enumerate}


\section{Introduction}

\subsection{Purpose of Document}
The primary purpose of this document is to capture and define the functional and non-functional requirements of the SweatSmart app. It will serve as a comprehensive reference that outlines what the application will need to do. It provides a roadmap for the development of the application. Additionally, this document serves as a means of communication between project stakeholders, ensuring all team members have a clear understanding of the project’s scope and requirements. By providing a place for clear specification of requirements and constraints, the SRS document also supports quality assurance efforts.

\subsection{Purpose of Project}
The purpose of “SweatSmart,” the AI-Powered Workout Planner, is to develop a sophisticated mobile application that harnesses the capabilities of artificial intelligence (AI) and machine learning (ML) to revolutionize how individuals plan and manage their fitness routines. This project’s main goal is to address the growing demand for purpose-built health and fitness solutions that cater to users with a diverse set of fitness goals, experience levels, preferences, and backgrounds.

\subsection{Goals}
The SweatSmart project is driven by a set of self-defined goals and objectives that shape its development and implementation. These goals reflect the project’s mission and the desired outcomes. The main project goals are listed as follows:

\subsubsection{Personalization}
The primary goal of the project is to deliver a highly personalized fitness experience to users of the SweatSmart application. Using AI and ML technologies, our team aims to create tailored and unique workout plans for each individual user. These plans should adapt to users’ individual progress, preferences, feedback, and evolving fitness goals.

\subsubsection{Guidance and Support}
The project aims to provide comprehensive guidance and support during workouts through the SweatSmart application, enhancing safety and effectiveness of the application during workouts.

\subsubsection{Progress Tracking}
SweatSmart aims to provide users with a wide range of tools to track their fitness progress over time. Enabling users to log completed workouts, record performance metrics, and visualize their fitness achievements.

\subsubsection{Simplicity and User-Centered Design}
Our team aims to deliver a user-friendly and intuitive experience with the SweatSmart application. The SweatSmart user interface will prioritize simplicity and ease-of-use so that users can navigate the app efficiently.

\subsubsection{Reliability}
This project strives to provide a robust and error-free application, operating without any critical errors leading to crashes or unexpected behavior.

\subsubsection{Efficient Code and Documentation}
Maintain clean and effective code while also providing comprehensive documentation.

\subsubsection{Budget Adherence}
Stay within the allotted budget of \$750 CAD.

\subsection{Stakeholders}

\subsubsection{Direct}
\begin{itemize}
  \item Beginners: Individuals with little to no experience in the fitness world.
  \item Fitness enthusiasts: Individuals who are already dedicated to maintaining an active and healthy lifestyle.
  \item Athletes: Competitive athletes seeking to optimize/improve their training regimens.
  \item Health-Focused Individuals: People with specific health objectives, such as weight management, stress reduction, or injury rehabilitation, who require bespoke fitness plans to achieve these goals.
\end{itemize}

\subsubsection{Indirect}
\begin{itemize}
  \item Fitness Industry Experts: Professionals in the fitness industry who may indirectly or directly influence the app’s content and application through their industry expertise and research.
  \item Fitness Equipment Manufacturers: Companies that produce fitness equipment may benefit from the application leading to an increase in users’ interest in fitness and related equipment.
  \item Local Fitness Facilities: Gyms, fitness centers, and local fitness trainers who may indirectly interact with users referred by the app or who engage with the fitness community fostered by the app.
\end{itemize}

\subsection{Assumptions}

\subsubsection{Project Assumptions}
\begin{itemize}
  \item The app will be available/maintained post the 8-month development period.
\end{itemize}

\subsubsection{User Assumptions}
\begin{itemize}
  \item Users are consistently updating the application to view the latest updates and changes.
  \item Users are assumed to have access to the internet at times when using certain features of the app.
  \item Users have basic knowledge of using mobile apps to navigate properly.
  \item Users use the app consistently for accurate fitness tracking and results.
\end{itemize}

\subsection{Off-the-shelf Solutions}

\subsubsection{EvolveAI}
EvolveAI is an application that combines artificial intelligence, industry-leading coaches, world-class athletes, and research to simulate a personal trainer and nutritionist in a digital era. Based on users’ dietary preferences and health status, the system focuses on users’ nutrition, on top of their workout routine, to help achieve their goals. The algorithm focuses on providing the right exercises and features videos and coaching cues to ensure proper form and technique. The system also integrates special features, including voice-to-text logging, adjustable workout intensity, and inbuilt stress index feature. [1]

\subsubsection{Fitbod}
Fitbod creates personalized training programs through an AI algorithm. It tracks users progress and uses that data to suggest changes in the users workout, such as weights and number of reps. It also recommends workouts based on muscle fatigue from previous sessions. This application also integrates fitbit, Apple Health, and other wearables. Lastly, it has an extensive library of exercises and high-quality video instructions to ensure proper form and injury prevention. [2]

\subsubsection{FitnessAI}
FitnessAi is another AI-powered fitness application that creates personalized workouts based on workout history, personal goals, and fitness levels. An interesting feature from this app is the use of 3D animation to show proper form and the targeted muscles of specific exercises. [3]


\section{System Description}

\subsection{Informal Description (if time permits)}

\subsection{System Context}

\subsubsection{Users}
At the heart of the system lie the users of the application. Users are a diverse group of individuals and interact with the SweatSmart app to access workout plans, track progress, and receive guidance on their fitness journey. Users are further categorized in section 2.4 “Stakeholders” so we will not go into detail here, but it is important to mention users when discussing context.

\subsubsection{Social Media Platforms}
We recognize how important social media is for the average person’s fitness journey. SweatSmart will need to utilize this excellent resource to help motivate our users and promote the app in the process. Users should be able to easily share progress updates and workouts they enjoyed to social media sites like Instagram, X (formerly Twitter), and Facebook.

\subsubsection{External Data Sources}
SweatSmart will need to harness the power and information of external data sources to enrich its functionality and provide a comprehensive data set for its ML model. Resources like fitness blogs, articles, research papers, nutritional databases, and much more will help the application deliver up-to-date health and fitness information.

\subsubsection{Fitness Equipment and Wearables (Future Integration)}
The current fitness market is full of wearable technologies that help track an extensive set of metrics to help people on their health and fitness journeys. Integration with wearable technology will allow for a more robust user experience with workouts seamlessly synchronized in real-time. Fitness equipment has also become more advanced, with features in equipment like free weights, stationary bikes, and treadmills, which enable users to accurately track their workouts. SweatSmart would see positive user engagement from integration with these devices.

\subsection{User Characteristics}
Users will have varying levels of fitness experience and knowledge. Users might be beginners at the gym who are looking to start their fitness journey, regular gym-goers who are looking to improve their workouts, or highly experienced fitness enthusiasts who would like to track their workouts. Users should have a basic understanding of how to use a smartphone or tablet. It is expected that users will already be familiar with downloading an application, opening an application, and navigating using a touch-screen device. It is important to note that this list is not comprehensive, but it helps illustrate the types of basic functions we expect our users to be familiar with before opening our application. Users will not require any formal training to be able to operate the system.

\subsection{Problem Description}
The fitness industry has seen a remarkable explosion in interest recently due to a variety of factors. However, many challenges still persist for individuals seeking effective fitness solutions despite the increase in popularity.

\begin{enumerate}
  \item One-Size-Fits-All Approach: Traditional fitness regimens often adopt this kind of approach, providing generic workout plans that do not consider the individuality of those seeking guidance.
  \item Lack of Personalization: Many apps offer limited means of personalization, resulting in user disengagement due to plans that fail to align with each individual’s needs and goals.
  \item Guidance Gaps: Users, namely beginners, frequently encounter gaps in guidance. This makes it challenging to understand proper exercise technique, safety measures, and what progress should look like.
  \item Motivation and Accountability: Staying accountable and motivated on a fitness journey is difficult without support and commitment to some sort of community.
  \item Complexity and Information Overload: Existing fitness apps and other solutions may overwhelm users with extensive data, metrics, and features, making the fitness experience too complex to be accessible.
\end{enumerate}

In summary, the current fitness landscape sees a lack of effective and personalized options for users, leading to less-than-ideal fitness outcomes and an overall lack of engagement.

\subsection{Use Cases/Scenarios}

\subsubsection{User Creates an Account}
Before a user can use the application, the user must create an account if an account has not been made yet.

\subsubsection{User Signs In}
If an account already exists and is associated with the user, the user must log in before using the application.

\subsubsection{Exercise is Added to Workout Plan}
A user is able to customize their workout plans by adding their own exercises to a tailored workout plan that they can follow.

\subsubsection{Exercise is Updated/Removed from Workout Plan}
Users can remove or update their workout plans by deleting exercises or an entire workout plan that they created.

\subsubsection{Live Workout is Started}
Users can start their live workout and add their weights, repetitions, and the number of sets to add detail to every exercise they perform.

\subsubsection{User Signs Out}
Users can sign out of their accounts.

\subsubsection{Update Personal Information}
Users will update personal information such as their weight, gender, age, etc. This will be used when a workout is generated by an AI.

\subsubsection{A Workout is Generated}
A workout is generated from an AI after user input.

\subsubsection{View Workout History}
Users can view their workout history once their live workouts have been tracked.

\subsubsection{Communicate with a Chatbot Coach}
Users can constantly chat with a virtual bot to get help related to fitness.


\section{Constraints}

\subsection{Input Data Constraints}

\subsubsection{User Data Validation}

All user input data related to personal information (e.g., age, weight, height, gender) must adhere to predefined formats and ranges.

\begin{itemize}
  \item Age: Must be an integer with a minimum value of at least 13 years.
  \item Weight: A positive number.
  \item Height: A positive number.
\end{itemize}

\subsubsection{Workout Session Data}

Users should be allowed to input a range for the duration of a workout session.

\subsubsection{Workout Intensity}

Fitness intensity of workouts should be categorized into predefined levels (e.g., Beginner, Intermediate, Advanced).

\subsubsection{Health and Safety Exercise Restrictions}

The system should restrict certain exercises or intensities for users with specific health conditions or preferences based on the input health profile.

\subsection{Timeline}

\subsubsection{Requirements}

The SRS is to be completed by October 6, 2023.

\subsubsection{Hazard Analysis}

The hazard analysis is to be completed by October 20, 2023.

\subsubsection{Verification and Validation (V&V)}

The VnV is to be completed by November 3, 2023.

\subsubsection{Proof of Concept}

The proof of concept is expected to be met by November 13, 2023.

\subsubsection{Design Documentation}

The Design documentation is to be completed by January 17, 2023.

\subsubsection{V&V Report}

The V&V report is to be completed by March 6, 2023.

\subsubsection{Final Deliverable}

The app is to be completed by March 18, 2023.

\subsection{Budget}

\subsubsection{Development Budget}

The total budget allocated for the entire development lifecycle of the application is \$750 CAD.

\subsubsection{Cloud Services}

The budget allocated for cloud services is \$100 CAD.

\section{Functional Requirements}

\subsection{User Account}

\parindent UA1: The app allows users to register an account using basic information.
\[Register(UserInfo) \rightarrow Account \]
\[ \forall UserInfo: (UserInfo \neq null) \Rightarrow (Account = true) \]\\

UA2: The app shall allow users to log in/log out of their account.
\[LogIn(UserCredentials) \rightarrow Session \]
\[ \forall UserCredentials: (UserCredentials \neq null) \Rightarrow (Session = true) \]
\[=LogOut(Session) \rightarrow \neg Session \]
\[ \forall Session: (Session = true) \Rightarrow (\neg Session = true) \]\\

UA3: The app shall allow users to create and update their profile on their account.
\[UpdateProfile(UserInput) \rightarrow UpdatedProfile \]
\[ \forall UserInput: (UserInput = true) \Rightarrow (UpdatedProfile = true) \]\\

UA4: The app must present the user with a terms and conditions agreement upon user registration.
\[PresentTerms(UserRegistration) \rightarrow TermsPresented \]
\[ \forall UserRegistration: (UserRegistration = true) \Rightarrow (TermsPresented = true) \]\\

UA5: The user must agree to the terms and conditions agreement to create an account.
\[AgreeToTerms(UserAgreement) \rightarrow AccountCreation \]
\[ \forall UserAgreement: (UserAgreement = true) \Rightarrow (AccountCreation = true) \]

\subsection{AI Workout Generation}

\parindent AI1: The system should generate personalized workout plans based on user input.
\[GenerateWorkoutPlan(UserInput) \rightarrow WorkoutPlan \]
\[ \forall UserInput: (UserInput \neq null) \Rightarrow (WorkoutPlan = true) \]\\

AI2: The system should alter and revise user workout plans.
\[AlterWorkoutPlan(WorkoutPlan, UserRequest) \rightarrow AlteredWorkoutPlan \]
\[ \forall (WorkoutPlan, UserRequest): (WorkoutPlan \land UserRequest = true) \Rightarrow\] \[(AlteredWorkoutPlan = true) \]\\

AI3: The system should recursively train the model based on user progress.
\[TrainModel(UserProgress, PreviousModel) \rightarrow UpdatedModel \]
\[ \forall (UserProgress, PreviousModel): (UserProgress \land PreviousModel = true) \Rightarrow\] \[(UpdatedModel = true) \]\\

AI4: The app shall have an AI chatbot that users can interact with.
\[InteractWithChatBot(UserInput) \rightarrow ChatBotResponse \]
\[ \forall UserInput: (UserInput \neq null) \Rightarrow (ChatBotResponse \neq null) \]\\

AI5: The app shall provide messages during a user's workout, giving the user tips for their specified exercises.
\[ProvideExerciseTips(UserWorkout) \rightarrow TipsProvided \]
\[ \forall UserWorkout: (UserWorkout = true) \Rightarrow (TipsProvided = true) \]

\subsection{User Workout Creation}
UC1: The app shall allow users to link their calendar to plan their workouts around their schedule.
\[LinkCalendar(User, Calendar) \rightarrow LinkedCalendar \]
\[ \forall (User, Calendar): (User \neq null \land Calendar \neq null) \Rightarrow (LinkedCalendar = true) \]
\\

UC2: The app shall allow users to rate their workouts after completion.
\[RateWorkout(User, Workout, Rating) \rightarrow WorkoutRating \]
\[ \forall (User, Workout, Rating):(User \neq null \land Workout \neq null \land Rating \in [1,5]) \Rightarrow (WorkoutRating = true) \]\\

UC3: The app shall allow users to design and save their own custom workouts tailored to their preferences.
\[CreateCustomWorkout(User, Preferences) \rightarrow CustomWorkout \]
\[ \forall (User, Preferences):(User \neq null \land Preferences \neq null) \Rightarrow (CustomWorkout = true) \]\\

UC4: The system should list all exercises.
\[ListExercises() \rightarrow ExerciseList \]
\[ \forall : () \Rightarrow (ExerciseList \neq null) \]

\subsection{Workout History}
WH1: The app shall display a user’s workout history.
\[DisplayHistory(User) \rightarrow History \]
\[ \forall User: (User \neq null) \Rightarrow (History \neq null) \]\\

WH2: The app saves a completed workout.
\[SaveCompletedWorkout(User, CompletedWorkout) \rightarrow SavedWorkout \]
\[ \forall (User, CompletedWorkout): (User \neq null \land CompletedWorkout = true) \Rightarrow (SavedWorkout = true) \]

\subsection{Database}
DB1: The app shall have a database to store user account information, workout plans, user feedback, and workout history.
\[StoreData(UserAccount, WorkoutPlan, UserFeedback, WorkoutHistory) \rightarrow StoredData \]
\[ \forall (UserAccount, WorkoutPlan, UserFeedback, WorkoutHistory):\] 
\[(UserAccount \neq null \land WorkoutPlan \neq null \land UserFeedback \neq null \land WorkoutHistory \neq null) \\
\Rightarrow\] \[(StoredData = true) \]

\subsection{Additional Features}
AF1: The app should allow users to begin/engage in a planned workout while tracking the workout live throughout.
\[TrackLiveWorkout(User, PlannedWorkout) \rightarrow LiveTracking \]
\[ \forall (User, PlannedWorkout): (User \neq null \land PlannedWorkout = true) \Rightarrow (LiveTracking = true) \]\\

AF2: The app shall be able to push notifications.
\[PushNotification(User, NotificationContent) \rightarrow NotificationPushed \]
\[ \forall (User, NotificationContent): (User \neq null \land NotificationContent \neq null) \Rightarrow\] \[(NotificationPushed = true) \]\\

AF3: The app allows users to start planned workouts and follow along live.
\[StartLiveWorkout(User, PlannedWorkout) \rightarrow LiveWorkoutSession \]
\[ \forall (User, PlannedWorkout): (User \neq null \land PlannedWorkout = true) \Rightarrow (LiveWorkoutSession = true) \]\\

AF4: The app allows users to download workouts for offline access.
\[DownloadWorkout(User, SelectedWorkout) \rightarrow OfflineWorkout \]
\[ \forall (User, SelectedWorkout): (User \neq null \land SelectedWorkout \neq null) \Rightarrow (OfflineWorkout = true) \]\\

AF5: The app shall allow users to share information from the application to social media.
\[ShareInfo(User, Information) \rightarrow InfoShared \]
\[ \forall (User, Information): (User \neq null \land Information \neq null) \Rightarrow (InfoShared = true) \]

\section{Non-functional Requirements}

\subsection{Look and Feel}

\subsubsection{Appearance Requirements}
\begin{itemize}
\item APR1: The app shall have an intuitive and user-friendly interface with minimalist design.
\item APR2: The app shall be able to adapt to a variety of screen sizes and aspect ratios.
\item APR3: The app shall provide visual feedback whenever a user action takes place.
\item APR4: The app shall load data at a reasonable time so there are no noticeable delays for users.
\item APR5: The app shall only support portrait mode.
\end{itemize}

\subsubsection{Styling Requirements}
\begin{itemize}
\item STR1: The app should have consistent styling and design throughout by using the same colour palette.
\item STR2: The application’s styling should be modern and similar to iOS apps.
\item STR3: The styling should be visually appealing and align with the app’s logo or theme.
\end{itemize}

\subsection{Usability and Human Requirements}

\subsubsection{Ease of Use Requirements}
\begin{itemize}
\item EUR1: The system shall be intuitive such that new users can quickly understand basic functions and commands by using intuitive icons.
\item EUR2: The user interface elements, terminology, and interactions should remain consistent throughout the application’s various pages.
\end{itemize}

\subsubsection{Personalization Requirements}
\begin{itemize}
\item PER1: The app will support local date/time formats based on the user’s device settings.
\item PER2: The app shall stay consistent with the personalization settings on the user’s device regarding display and font.
\end{itemize}

\subsubsection{Learning Requirements}
\begin{itemize}
\item LER1: Users shall not require external resources to navigate the application; everything they need to know to use the app should be taught in-app.
\end{itemize}

\subsubsection{Understandability and Politeness Requirements}
\begin{itemize}
\item UPR1: The system should provide a tutorial for first-time users.
\item UPR2: The application shall contain an FAQ to answer expected questions about the application.
\end{itemize}

\subsubsection{Accessibility Requirements}
\begin{itemize}
\item ACR1: The system will follow Apple’s design guidelines regarding accessible applications.
\end{itemize}

\subsection{Performance Requirements}
\begin{itemize}
\item PR1: The system shall be capable of handling a minimum of 1000 simultaneous users and processing 200 transactions per minute while maintaining performance standards.
\item PR2: The system shall achieve a minimum of 99.9% availability, ensuring services are accessible to users at all times, excluding scheduled maintenance windows.
\item PR3: The system shall support a data throughput of at least 200 Mbps to ensure data transmission between the server and users is seamless and efficient under peak usage.
\item PR4: The system shall deliver a response time of no more than 2 seconds for 99% of the transaction processed under standard operating conditions.
\end{itemize}

\subsection{Operational and Environment Requirements}
\begin{itemize}
\item OER1: The system shall be able to be used in a gym/workout facility or home workout room.
\item OER2: The system shall be able to run on an iOS phone.
\item OER3: The system shall allow the audio output to be transmitted through the device’s speakers or a connected Bluetooth device.
\item OER4: The system should connect to the internet for back-end services.
\end{itemize}

\subsection{Maintainability and Support Requirements}
\begin{itemize}
\item MSR1: The system shall automatically detect, log, and report any errors or system failures to the development and support team without requiring user intervention.
\item MSR2: The system shall incorporate an accessible user feedback mechanism to enable users to easily report issues.
\item MSR3: The system shall support periodic, backward-compatible software updates, notifying users of availability and enabling convenient download and installation while providing information about the changes.
\end{itemize}

\subsection{Compliance Requirements}
\begin{itemize}
\item COMR1: The application will adhere to all data protection laws in the region(s) it operates within.
\item COMR2: The informed consent of users shall be obtained before collecting and processing their personal data.
\item COMR3: Robust security measures will be put in place to safeguard user data.
\item COMR4: The application will be transparent about its data collection.
\item COMR5: The app shall always recommend safe and proven fitness guidance and information.
\item COMR6: The system must state that the application is not responsible for any injuries or accidents resulting from its use.
\end{itemize}

\subsection{Security Requirements}
\begin{itemize}
\item SECR1: The app shall enforce strong password restrictions to ensure users are properly protecting their own data.
\item SECR2: Users shall authenticate themselves with their credentials securely.
\item SECR3: Sensitive data shall be restricted from the users of the app.
\item SECR4: Users will only have access to data and features for which they are authorized.
\item SECR5: Data between the app and the server should be encrypted when an API call is made.
\item SECR6: User data will be encrypted to ensure it is transmitted securely through a server.
\end{itemize}

\section{Likely Changes}

LC1: The application may need to adapt to future versions of iOS to ensure compatibility and leverage new features.\\
LC2: The integration of new APIs or third-party services to enhance functionality, such as adding a new payment gateway or integrating a new workout tracking device.\\
LC3: Refinements in the UI/UX based on user feedback and usability testing may be required post-launch.\\
LC4: The addition of new features or screens to accommodate evolving user needs and market trends, such as incorporating a new workout trend or social sharing capabilities.\\
LC5: As the user base grows, enhancing the backend infrastructure to manage increased load, reducing latency, and ensuring smooth user experiences.\\
LC6: Changes in data protection laws or health app regulations may require adjustments in data handling and user privacy features.\\
LC7: Inclusion of new features or modification of existing ones to stay competitive and meet user demands.

\section{Unlikely Changes}

UL1: The basic functionality and core features of the app, such as tracking workouts and providing workout plans, are fundamental and unlikely to change drastically.\\
UL2: The primary target audience, being fitness enthusiasts, gym-goers, is unlikely to change.\\
UL3: The platform choice (iOS) given that the styling, design, and functionalities, are deeply integrated with iOS features.\\
UL4: The programming languages and core technologies used in development are unlikely to change unless a significant technological shift happens.


\section{Traceability Matrices}
Table 1 and Table 2 show the dependencies between the functional requirements. A requirement in the first column that is dependent on a requirement in the first row is marked by an 'X'.


\afterpage{
\begin{landscape}

\begin{table}[h!]
\centering
\begin{tabular}{|c|c|c|c|c|c|c|c|c|c|c|c|c|c|c|c|c|c|}
 \hline
   & AI1 & AI2 & AI3 & AI4 & AI5 & UC1 & UC2 & UC3 & UC4 & WH1 & WH2 & DB1 & AF1 & AF2 & AF3 & AF4 & AF5\\
 \hline
 UA1 & & & & & & & & & & & & & & & & & & 
 \hline
 UA2 & & & & & & & & & & & & & & & & & & 
 \hline
 UA3 & & & & & & & & & & & & X & & & & & & 
 \hline
 UA4 & & & & & & & & & & & & & & & & & & 
 \hline
 UA5 & & & & & & & & & & & & & & & & & & 
 \hline
 AI1 & & & & & & & & & & & & & & & & & & 
 \hline
 AI2 & X & & & & & & & & & & & & & & & & & 
 \hline
 AI3 & & & & & & & X & & & & & X & & & & & & 
 \hline
 AI4 & & & & & & & & & & & & & & & & & & 
 \hline
 AI5 & & & & & & & & & & & & & & & & & & 
 \hline
 UC1 & & & & & & & & & & & & & & & & & & 
 \hline
 UC2 & & & & & & & & & & & & & & & X & & & 
 \hline
 UC3 & & & & & & & & & & & & X & & & & & & 
 \hline
 UC4 & & & & & & & & & & & & & & & & & & 
 \hline
 WH1 & & & & & & & & & & & & & & & & & & 
 \hline
 WH2 & & & & & & & & & & & & X & & & & & & 
 \hline
 DB1 & & & & & & & & & & & & & & & & & & 
 \hline
 AF1 & & & & & & X & & & & & & & & & & & & 
 \hline
 AF2 & & & & & & & & & & & & & & & & & & 
 \hline
 AF3 & & & & & & & & & & & & & & & & & & 
 \hline
 AF4 & & & & & & & & & & & & & & & & & & 
 \hline
 AF5 & & & & & & & & & & & & & & & & & & 
\hline
\end{tabular}
\caption{Traceability Matrix Showing the Mappings Between Functional Requirements Cont.}
\label{Table:A_trace}
\end{table}

\end{landscape}
}

\begin{table}
\centering
\begin{tabular}{|c|c|c|c|c|c|}
 \hline
  & UA1 & UA2 & UA3 & UA4 & UA5 \\
 \hline
 UA1 & & & & X &\\
 \hline
 UA2 & X & & & & \\
 \hline
 UA3 & & X & & & \\
 \hline
 UA4 & & & & & \\
 \hline
 UA5 & X & & & & \\
 \hline
 AI1 & & & & & \\
 \hline
 AI2 & & & & & \\
 \hline
 AI3 & & & & & \\
 \hline
 AI4 & & & & & \\
 \hline
 AI5 & & & & & \\
 \hline
 UC1 & & & & & \\
 \hline
 UC2 & & & & & \\
 \hline
 UC3 & & & & & \\
 \hline
 UC4 & & & & &  \\
 \hline
 WH1 & & & & &  \\
 \hline
 WH2 & & & & & \\
 \hline
 DB1 & & & & &  \\
 \hline
 AF1 & & & & & \\
 \hline
 AF2 & & & & & \\
 \hline
 AF3 & & & & & \\
 \hline
 AF4 & & & & & \\
 \hline
 AF5 & & & & & \\
\hline
\end{tabular}
\caption{Traceability Matrix Showing the Mappings Between Functional Requirements}
\end{table}


\section{Development Plan}

The development of the fitness application will follow a phased approach to ensure efficient progress and quality assurance. The order of requirements implementation is as follows:

\subsection{Phase 1: Basic Functionality}

In this initial phase, we will focus on implementing the core features required for the app's basic functionality. This includes:

\begin{itemize}
\item User account management (UA1 - UA5)
\item AI Workout Generation (AI1 - AI5)
\item User Workout Creation (UC1 - UC4)
\end{itemize}

\subsection{Phase 2: User Experience Enhancement}

During this phase, we will concentrate on improving the user experience by addressing usability and appearance requirements. Key tasks include:

\begin{itemize}
\item Implementing appearance and styling requirements (APR1 - APR5, STR1 - STR3)
\item Enhancing usability and personalization (EUR1 - EUR2, PER1 - PER2)
\item Implementing learning, understandability, and politeness requirements (LER1, UPR1 - UPR2)
\end{itemize}

\subsection{Phase 3: Performance and Accessibility}

In this phase, we will ensure that the app meets performance, accessibility, and compatibility requirements. This includes:

\begin{itemize}
\item Addressing performance requirements (PR1 - PR4)
\item Ensuring operational and environmental compatibility (OER1 - OER4)
\item Implementing accessibility guidelines (ACR1)
\end{itemize}

\subsection{Phase 4: Security and Compliance}

The final phase will focus on security and compliance with relevant regulations. Key tasks involve:

\begin{itemize}
\item Implementing security measures (SECR1 - SECR6)
\item Ensuring compliance with data protection laws and user consent (COMR1 - COMR4)
\item Addressing safety and health recommendations (COMR5 - COMR6)
\end{itemize}

\subsection{Phase 5: Likely Changes and Post-launch Refinements}

After the initial development and launch, we will continuously monitor user feedback and market trends. We will be prepared to implement likely changes (LC1 - LC7) and make refinements as needed to enhance the app's features and performance.

\subsection{Phase 6: Unlikely Changes and Long-Term Maintenance}

While the core features (UL1 - UL4) are unlikely to change significantly, we will remain committed to maintaining the app, ensuring its compatibility with future iOS versions, and providing ongoing support and updates. 


\newpage

\bibliographystyle {plainnat}
\bibliography {../../refs/References}



\newpage{}
\section*{Appendix --- Reflection}

The information in this section will be used to evaluate the team members on the
graduate attribute of Lifelong Learning.  Please answer the following questions:

\begin{enumerate}
  \item What knowledge and skills will the team collectively need to acquire to
  successfully complete this capstone project?  Examples of possible knowledge
  to acquire include domain specific knowledge from the domain of your
  application, or software engineering knowledge, mechatronics knowledge or
  computer science knowledge.  Skills may be related to technology, or writing,
  or presentation, or team management, etc.  You should look to identify at
  least one item for each team member.\\

Fitness and Health Domain:\\
Understanding the fitness and health domain, including workout planning, nutrition, and exercise physiology, is essential for designing effective and personalized fitness solutions.

Machine Learning:\\
Our team will need to gain expertise in ML and AI methodologies in order to develop SweatSmart. To train the AI model, we first need to collect quantitative and qualitative data. Using the data, we can build and train the algorithm to generate personalized workouts.

Mobile App Development:\\
Mastery of mobile app development for iOS will be necessary for our team. We plan on using React to develop the front-end and .NET for the back end. All team members will need to learn and/or refine knowledge in these development tools for this project to be successful.

UI/UX Design:\\
Understanding key principles of user interface and user experience design to create intuitive, visually appealing, and user-friendly interfaces will be important. We will use Figma to visualize pages of our app, making it more efficient to code.

Database Management:\\
Skills in database design, data storage, and retrieval are needed to manage user accounts, workout plans, and progress data securely.

  \item For each of the knowledge areas and skills identified in the previous
  question, what are at least two approaches to acquiring the knowledge or
  mastering the skill?  Of the identified approaches, which will each team
  member pursue, and why did they make this choice?


\textbf{Fitness Domain Research:}\\
Team Members: Sam and Jonny

To ensure a comprehensive understanding of fitness practices, we will conduct research by:
\begin{itemize}
    \item Reading at least 5 peer-reviewed articles on the latest fitness practices.
    \item Exploring 10 or more fitness-related blogs each.
    \item Analyzing 3 or more existing fitness applications/websites each.
\end{itemize}

This extensive research will provide us with diverse insights into the current fitness landscape, allowing us to make informed decisions during app development.

\textbf{Consultation with Fitness Experts:}\\
Team Members: Sophie and Daniel

We will consult with fitness experts, including those at McMaster University's kinesiology department, McMaster personal trainers, and potentially faculty members from the Physiotherapy department. This step is crucial for gaining insights and guidance from primary sources who are experts in the field of fitness.

\textbf{Machine Learning Skill Development:}\\
To gain proficiency in machine learning, we will:
\begin{itemize}
    \item Utilize online resources for learning how to create and train AI algorithms.
    \item Seek assistance from professors and industry experts in the field.
\end{itemize}

\textbf{Mobile App Development:}\\

To enhance our mobile app development skills, we will:
\begin{itemize}
    \item Develop a simple, bare-bones app to practice fundamental functions using React and .Net. (Team Members: Sophie Fillion, Sam McDonald). This will be suefil practice to ensure we understand all basic functions and practices for developing a mobile application using React and .Net. 
    \item Practice writing code for mobile platforms (Team Members: Jonny and Daniel). We are already familiar with React and .Net for mobile development, so we will write smaller practice sections and assist Sophie and Sam in developing their basic apps, serving as mentors to ensure a strong understanding of these frameworks.
\end{itemize}


\textbf{UI/UX Design Skill Development:}\\
To improve our UI/UX design skills, we will:
\begin{itemize}
    \item Create design mockups (Team Members: Jonny and Daniel). Justification: Collaborative tutorials and learning from fellow team members are highly effective ways to enhance skills. Jonny, with expertise in UI/UX design, will provide tutorials, allowing the team to align on design principles before commencing app development.
\end{itemize}

\textbf{Database Management Skill Development:}\\
To strengthen our database management skills, we will:
\begin{itemize}
    \item Review class notes on SQL queries from our databases class.
    \item Utilize online resources for additional learning.
\end{itemize}

\end{enumerate}

\end{document}