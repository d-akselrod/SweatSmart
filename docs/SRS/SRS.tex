% THIS DOCUMENT IS TAILORED TO REQUIREMENTS FOR SCIENTIFIC COMPUTING.  IT SHOULDN'T
% BE USED FOR NON-SCIENTIFIC COMPUTING PROJECTS
\documentclass[12pt]{article}

\usepackage{amsmath, mathtools}
\usepackage{amsfonts}
\usepackage{amssymb}
\usepackage{graphicx}
\usepackage{colortbl}
\usepackage{xr}
\usepackage{hyperref}
\usepackage{longtable}
\usepackage{xfrac}
\usepackage{tabularx}
\usepackage{float}
\usepackage{siunitx}
\usepackage{booktabs}
\usepackage{caption}
\usepackage{pdflscape}
\usepackage{afterpage}

\usepackage[round]{natbib}

%\usepackage{refcheck}

\hypersetup{
    bookmarks=true,         % show bookmarks bar?
      colorlinks=true,       % false: boxed links; true: colored links
    linkcolor=red,          % color of internal links (change box color with linkbordercolor)
    citecolor=green,        % color of links to bibliography
    filecolor=magenta,      % color of file links
    urlcolor=cyan           % color of external links
}

\input{../Comments}
%% Common Parts

\newcommand{\progname}{SFWRENG 4G06} % PUT YOUR PROGRAM NAME HERE
\newcommand{\authname}{Team \#7, Team FAAM, SweatSmart
\\
\\ Daniel Akselrod
\\ Jonathan Avraham
\\ Sophie Fillion
\\ Sam McDonald} % AUTHOR NAMES                  

\usepackage{hyperref}
\hypersetup{colorlinks=true, linkcolor=blue, citecolor=blue, filecolor=blue,
    urlcolor=blue, unicode=false}
\urlstyle{same}
                                


% For easy change of table widths
\newcommand{\colZwidth}{1.0\textwidth}
\newcommand{\colAwidth}{0.13\textwidth}
\newcommand{\colBwidth}{0.82\textwidth}
\newcommand{\colCwidth}{0.1\textwidth}
\newcommand{\colDwidth}{0.05\textwidth}
\newcommand{\colEwidth}{0.8\textwidth}
\newcommand{\colFwidth}{0.17\textwidth}
\newcommand{\colGwidth}{0.5\textwidth}
\newcommand{\colHwidth}{0.28\textwidth}

% Used so that cross-references have a meaningful prefix
\newcounter{defnum} %Definition Number
\newcommand{\dthedefnum}{GD\thedefnum}
\newcommand{\dref}[1]{GD\ref{#1}}
\newcounter{datadefnum} %Datadefinition Number
\newcommand{\ddthedatadefnum}{DD\thedatadefnum}
\newcommand{\ddref}[1]{DD\ref{#1}}
\newcounter{theorynum} %Theory Number
\newcommand{\tthetheorynum}{TM\thetheorynum}
\newcommand{\tref}[1]{TM\ref{#1}}
\newcounter{tablenum} %Table Number
\newcommand{\tbthetablenum}{TB\thetablenum}
\newcommand{\tbref}[1]{TB\ref{#1}}
\newcounter{assumpnum} %Assumption Number
\newcommand{\atheassumpnum}{A\theassumpnum}
\newcommand{\aref}[1]{A\ref{#1}}
\newcounter{goalnum} %Goal Number
\newcommand{\gthegoalnum}{GS\thegoalnum}
\newcommand{\gsref}[1]{GS\ref{#1}}
\newcounter{instnum} %Instance Number
\newcommand{\itheinstnum}{IM\theinstnum}
\newcommand{\iref}[1]{IM\ref{#1}}
\newcounter{reqnum} %Requirement Number
\newcommand{\rthereqnum}{R\thereqnum}
\newcommand{\rref}[1]{R\ref{#1}}
\newcounter{nfrnum} %NFR Number
\newcommand{\rthenfrnum}{NFR\thenfrnum}
\newcommand{\nfrref}[1]{NFR\ref{#1}}
\newcounter{lcnum} %Likely change number
\newcommand{\lthelcnum}{LC\thelcnum}
\newcommand{\lcref}[1]{LC\ref{#1}}

\usepackage{fullpage}

\newcommand{\deftheory}[9][Not Applicable]
{
\newpage
\noindent \rule{\textwidth}{0.5mm}

\paragraph{RefName: } \textbf{#2} \phantomsection 
\label{#2}

\paragraph{Label:} #3

\noindent \rule{\textwidth}{0.5mm}

\paragraph{Equation:}

#4

\paragraph{Description:}

#5

\paragraph{Notes:}

#6

\paragraph{Source:}

#7

\paragraph{Ref.\ By:}

#8

\paragraph{Preconditions for \hyperref[#2]{#2}:}
\label{#2_precond}

#9

\paragraph{Derivation for \hyperref[#2]{#2}:}
\label{#2_deriv}

#1

\noindent \rule{\textwidth}{0.5mm}

}

\begin{document}

\title{Software Requirements Specification\\ \progname} 
\author{\authname}
\date{\today}
	
\maketitle

~\newpage

\pagenumbering{roman}

\tableofcontents

~\newpage

\section*{Revision History}

\begin{table}[hp]
		\centering
		\begin{tabularx}{\textwidth}{lllX}
			\toprule
			\textbf{Revision Version} & \textbf{Date} & \textbf{Developer(s)} & \textbf{Change}\\
			\midrule
			0 & Oct 6, 2023 & Sophie, Daniel, Sam, Jonathan & First draft\\
            \hline
            0.1 & Oct 20, 2023 & Sophie, Jonathan & Updated safety requirements that mitigate system hazards\\
            \hline
            1.0 & Apr 3, 2024 & Sam & Updated the document to align with new project direction and final submitted code.\\
			\bottomrule
		\end{tabularx}
	\end{table}


~\newpage

\pagenumbering{arabic}

\section{Reference Material}

\subsection{Terminology}
\begin{enumerate}
    \item Rep - short for repetition 
\end{enumerate}


\section{Introduction}

\subsection{Purpose of Document}
The primary purpose of this document is to capture and define the functional and non-functional requirements of the SweatSmart app. It will serve as a comprehensive reference that outlines what the application will need to do. It provides a roadmap for the development of the application. Additionally, this document serves as a means of communication between project stakeholders, ensuring all team members have a clear understanding of the project’s scope and requirements. By providing a place for clear specification of requirements and constraints, the SRS document also supports quality assurance efforts.

\subsection{Purpose of Project}
The purpose of this document is to outline the functional and non-functional requirements for the SweatSmart app, focusing on evidence-based workout guidance. This document will act as a detailed guide for the app's development and an informative tool for stakeholders, ensuring a shared understanding of the project's scope. It will detail specifications and constraints, underpinning the quality assurance process and the project's commitment to an evidence-based approach, simplicity, and user accessibility.

\subsection{Goals}
The vision of SweatSmart is anchored in a set of deliberate objectives that guide its creation and rollout. These goals embody the app's essence and the outcomes it seeks to realize.

\begin{itemize}
    \item \textbf{Evidence-Based Personalization:} We aim to set ourselves apart from other fitness solutions by offering workout plans founded on scientific evidence, tailored to each user's developmental journey and fitness aspirations.
    \item \textbf{Guidance and Support:} We will deliver thorough workout guidance to ensure user safety and enhance the effectiveness of each session.
    \item \textbf{Progress Tracking:} The app will enable users to log workouts, both app-generated and self-made, view workout history, and repeat saved workouts. This is an important goal as progress tracking and history helps people stay on top of their goals and ensures they do not overexert themselves.
    \item \textbf{Simplicity and User-Centered Design:} SweatSmart is dedicated to providing an intuitively navigable interface, emphasizing straightforwardness to facilitate a straightforward user experience for people with limited technological experience.
    \item \textbf{Reliability:} Our aim is to develop a dependable application, free from critical malfunctions that could impede the user's progress.
    \item \textbf{Efficient Code and Documentation:} We prioritize maintaining best practices when it comes to coding standards and comprehensive documentation to underpin the app's structure and functionality.
    \item \textbf{Budget Adherence:} Stay within the allotted budget (Section 4.3).
\end{itemize}

\subsection{Stakeholders}

\subsubsection{Direct}
\begin{itemize}
  \item Beginners: Individuals with little to no experience in the fitness world and want a simple introduction to it.
  \item Fitness enthusiasts: Individuals who are already dedicated to maintaining an active and healthy lifestyle and seek a workout logger and/or new workout ideas to keep them interested in fitness.
  \item Athletes: Competitive athletes seeking to log and track their workout progress.
\end{itemize}

\subsubsection{Indirect}
\begin{itemize}
  \item Fitness Industry Experts: Professionals in the fitness industry, including our supervisors, Stuart Philips and Bradley Currier, who play a key role in informing the decision-making behind the workout generation algorithm and the app as a whole.
  \item Fitness Equipment Manufacturers: Companies that produce fitness equipment may benefit from the application leading to an increase in users’ interest in fitness and related equipment.
  \item Local Fitness Facilities: Gyms, fitness centers, and local fitness trainers who may indirectly interact with users referred by the app or who engage with the fitness community fostered by the app.
\end{itemize}

\subsection{Assumptions}

\subsubsection{Project Assumptions}
\begin{itemize}
  \item The app will be available/maintained post the 8-month development period.
\end{itemize}

\subsubsection{User Assumptions}
\begin{itemize}
  \item Users have varying levels of knowledge about fitness, engaging with the app in a variety of ways as a result.
  \item Users are assumed to have access to the internet at times when using certain features of the app.
  \item Users have basic knowledge of using mobile apps to navigate properly.
  \item Users use the app consistently for accurate fitness tracking and results.
  \item Users have access to workout equipment or are interested in body-weight exercises. 
\end{itemize}

\subsection{Off-the-shelf Solutions}

\subsubsection{EvolveAI}
EvolveAI is an application that combines artificial intelligence, industry-leading coaches, world-class athletes, and research to simulate a personal trainer and nutritionist in a digital era. Based on users’ dietary preferences and health status, the system focuses on users’ nutrition, on top of their workout routine, to help achieve their goals. The algorithm focuses on providing the right exercises and features videos and coaching cues to ensure proper form and technique. The system also integrates special features, including voice-to-text logging, adjustable workout intensity, and inbuilt stress index feature. [1]

\subsubsection{Fitbod}
Fitbod creates personalized training programs through an AI algorithm. It tracks users progress and uses that data to suggest changes in the users workout, such as weights and number of reps. It also recommends workouts based on muscle fatigue from previous sessions. This application also integrates fitbit, Apple Health, and other wearables. Lastly, it has an extensive library of exercises and high-quality video instructions to ensure proper form and injury prevention. [2]

\subsubsection{FitnessAI}
FitnessAi is another AI-powered fitness application that creates personalized workouts based on workout history, personal goals, and fitness levels. An interesting feature from this app is the use of 3D animation to show proper form and the targeted muscles of specific exercises. [3]

\section{System Description}

\subsection{System Context}

\subsubsection{Users}
At the heart of the system lie the users of the application. Users are a diverse group of individuals and interact with the SweatSmart app to access workout plans, track progress, and receive guidance on their fitness journey. Users are further categorized in section 2.4 “Stakeholders” and in section 3.2 "User Characteristics", so we will not go into detail here, but it is important to mention users when discussing context.

\subsubsection{Social Media Platforms (Future Consideration)}
We recognize how important social media is for the average person’s fitness journey. SweatSmart will need to utilize this excellent resource to help motivate our users and promote the app in the process. Users should be able to easily share progress updates and workouts they enjoyed to social media sites like Instagram, X (formerly Twitter), and Facebook.

\subsubsection{Kinesiology Experts}
Our team will harness the knowledge behind our supervisors' research into fitness and resistance training to inform SweatSmart's evidence-based approach to workout generation and tracking. These experts will need to be consulted throughout the development process to ensure every aspect of our app makes sense in the context of exercise science.

\subsubsection{Fitness Equipment and Wearables (Future Integration)}
The current fitness market is full of wearable technologies that help track an extensive set of metrics to help people on their health and fitness journeys. Integration with wearable technology will allow for a more robust user experience with workouts seamlessly synchronized in real-time. Fitness equipment has also become more advanced, with features in equipment like free weights, stationary bikes, and treadmills, which enable users to accurately track their workouts. SweatSmart would see positive user engagement from integration with these devices.

\subsection{User Characteristics}
Users will have varying levels of fitness experience and knowledge. Users might be beginners at the gym who are looking to start their fitness journey, regular gym-goers who are looking to easily track, update, or change up their workouts, or highly experienced fitness enthusiasts who would like to track their workouts. Users should have a basic understanding of how to use a smartphone or tablet. It is expected that users will already be familiar with downloading an application, opening an application, and navigating using a touch-screen device. It is important to note that this list is not comprehensive, but it helps illustrate the types of basic functions we expect our users to be familiar with before opening our application. Users will not require any formal training to be able to operate the system.

\subsection{Problem Description}

In our current fast-paced world, it has become increasingly difficult for people of all fitness levels to maintain a healthy and active lifestyle without investing significant time into creating a plan and routine. It does not help that self-proclaimed gym experts and "fitfluencers" have popped up on social media to over complicate the world of fitness even further, usually trying to sell expensive "solutions". Consequently, many individuals struggle to establish and follow personalized workout plans that match their experience, preferences, goals, and tight schedules. Market-available options often fail to provide adaptive and evolving workouts, leading to stagnation, reduced motivation, and boredom. Addressing these challenges, our team proposes the development of an evidence-based workout planner and logger app to assist individuals along their fitness journey. Our app seeks to simplify the fitness landscape for individuals trying to begin their fitness journey, emphasizing a strong evidence-base and an accessible design. 

\subsection{Use Cases/Scenarios}

\subsubsection{User Creates an Account}
Before a user can use the application, the user must create an account if an account has not been made yet.

\subsubsection{User Signs In}
If an account already exists and is associated with the user, the user must log in before using the application.

\subsubsection{Exercise is Added to Workout Plan}
A user is able to customize their workout plans by adding their own exercises to a tailored workout plan that they can follow.

\subsubsection{Exercise is Updated/Removed from Workout Plan}
Users can remove or update their workout plans by deleting exercises or an entire workout plan that they created.

\subsubsection{Live Workout is Started}
Users can start their live workout and add their weights, repetitions, and the number of sets to add detail to every exercise they perform.

\subsubsection{User Signs Out}
Users can sign out of their accounts.

\subsubsection{Update Personal Information}
Users will update personal information such as their fitness goals, fitness experience, or available equipment etc. This will be used when a workout is generated by the evidence-based algorithm.

\subsubsection{A Workout is Generated}
A workout is generated from our evidence-based algorithm.

\subsubsection{View Workout History}
Users can view their workout history once their live workouts have been tracked and finished.

\subsubsection{Communicate with a Chatbot Coach}
Users can chat with a virtual bot to ask any fitness-related questions.

\subsubsection{Users can Log their own Workouts}
Users can use the app to log workouts they came up with themselves.

\section{Constraints}

\subsection{Input Data Constraints}

\subsubsection{User Data Validation}

All user input data related to personal information must adhere to predefined formats and ranges.

\begin{itemize}
  \item Age: Must be an integer with a minimum value of at least 13 years.
  \item Experience Level: Must be either "Beginner", "Intermediate", or "Advanced".
  \item Available Equipment: Must be either "None", "Dumbbells only", or "Full". 
  \item Weekly Workout Frequency: Must be an integer between 1 and 7.
  \item Maximum Workout Duration: Must be an integer great than 15. 
  \item Fitness Goal: Must be either "General Health", "Strength", or "Endurance". 
\end{itemize}

\subsubsection{Health and Safety Exercise Considerations}

The system should contain a terms and conditions section to ensure users understand the safety risks that come with exercising and understand that our team takes no responsibility for user injury.

\subsection{Timeline}

\subsubsection{Requirements}

The SRS is to be completed by October 6, 2023.

\subsubsection{Hazard Analysis}

The hazard analysis is to be completed by October 20, 2023.

\subsubsection{Verification and Validation (V\&V)}

The VnV is to be completed by November 3, 2023.

\subsubsection{Proof of Concept}

The proof of concept is expected to be met by November 13, 2023.

\subsubsection{Design Documentation}

The Design documentation is to be completed by January 17, 2024.

\subsubsection{V\&V Report}

The V\&V report is to be completed by March 6, 2024.

\subsubsection{Final Deliverable}

The app is to be completed by March 18, 2024. The EXPO poster and video shall be completed by April 1, 2024. The final documentation is to be completed by April 4, 2024. 

\subsection{Budget}

\subsubsection{Development Budget}

The total budget allocated for the entire development lifecycle of the application is \$750 CAD.

\subsubsection{Cloud Services}

The budget allocated for cloud services is \$100 CAD.

\section{Functional Requirements}

\subsection{User Account}

\parindent UA1: The app allows users to register an account using basic information.
\[Register(UserInfo) \rightarrow Account \]
\[ \forall UserInfo: (UserInfo \neq null) \Rightarrow (Account = true) \]\\
\textit{Rationale:} A user registration system is essential for a personalized experience. By collecting basic information, the app can create unique profiles for security, customization, and content personalization, laying the groundwork for all future interactions.
\\

UA2: The app shall allow users to log in/log out of their account.
\[LogIn(UserCredentials) \rightarrow Session \]
\[ \forall UserCredentials: (UserCredentials \neq null) \Rightarrow (Session = true) \]
\[=LogOut(Session) \rightarrow \neg Session \]
\[ \forall Session: (Session = true) \Rightarrow (\neg Session = true) \]\\
\textit{Rationale:} The login/logout functionality is critical for user privacy and data security. It ensures that personal and sensitive data remain confidential and accessible only to the authenticated user, safeguarding against unauthorized access.
\\

UA3: The app shall allow users to create and update their profile on their account.
\[UpdateProfile(UserInput) \rightarrow UpdatedProfile \]
\[ \forall UserInput: (UserInput = true) \Rightarrow (UpdatedProfile = true) \]\\
\textit{Rationale:} Users' ability to create and modify their profiles allows for a tailored app experience that can evolve with the user’s changing fitness goals and preferences, ensuring ongoing personalization and engagement.
\\

UA4: The app must present the user with a terms and conditions agreement upon user registration.
\[PresentTerms(UserRegistration) \rightarrow TermsPresented \]
\[ \forall UserRegistration: (UserRegistration = true) \Rightarrow (TermsPresented = true) \]\\
\textit{Rationale:} Presenting terms and conditions during registration fulfills legal obligations and informs users of their rights and responsibilities, ensuring transparency and protecting the developers legally.
\\

UA5: The user must agree to the terms and conditions agreement to create an account.
\[AgreeToTerms(UserAgreement) \rightarrow AccountCreation \]
\[ \forall UserAgreement: (UserAgreement = true) \Rightarrow (AccountCreation = true) \]
\textit{Rationale:} Requiring agreement to terms and conditions ensures a clear contractual relationship between the user and the app, which is pivotal for informed consent and compliance with data protection regulations.
\\

\subsection{Evidence-Based Workout Generation}

\parindent EB1: The system should generate workout plans based on established exercise science principles.
\[GenerateWorkoutPlan(ScienceBasedInput) \rightarrow WorkoutPlan \]
\[ \forall ScienceBasedInput: (ScienceBasedInput \neq null) \Rightarrow (WorkoutPlan = true) \]
\textit{Rationale:} Leveraging established exercise science principles to generate workout plans ensures that the recommendations are credible, safe, and effective. This approach aligns the application with scientific standards, enhancing the trust and reliability perceived by users.
\\

EB2: The system should allow users to adjust workout plans according to their own preferences.
\[AdjustWorkoutPlan(WorkoutPlan, Preferences) \rightarrow AdjustedWorkoutPlan \]
\[ \forall (WorkoutPlan, Preferences): (WorkoutPlan \land Preferences = true) \Rightarrow\] \[(AdjustedWorkoutPlan = true) \]
\textit{Rationale:} Allowing users to adjust workout plans based on their preferences empowers them to tailor their fitness journey to their individual needs, increasing the likelihood of continued use and adherence to the workout regimen.
\\

EB3: The system should provide interactive guidance that aligns with evidence-based fitness protocols.
\[ProvideInteractiveGuidance(UserInteraction) \rightarrow GuidanceResponse \]
\[ \forall UserInteraction: (UserInteraction \neq null) \Rightarrow (GuidanceResponse \neq null) \]
\textit{Rationale:} Providing interactive guidance helps to simulate a personal training experience, offering users real-time, evidence-based feedback that can help improve their workout effectiveness and safety.
\\

EB4: The system should offer science-backed tips and instructions during workouts for optimal exercise execution.
\[ProvideExerciseInstructions(UserWorkout) \rightarrow InstructionsProvided \]
\[ \forall UserWorkout: (UserWorkout = true) \Rightarrow (InstructionsProvided = true) \]
\textit{Rationale:} Science-backed tips and instructions during workouts help users perform exercises correctly, maximizing benefits and minimizing the risk of injury. This educative aspect of the app promotes better exercise habits and informs users about the rationale behind each exercise.

\subsection{User Workout Creation}
UC1: The system shall generate a workout plan based on the user's desired weekly workout frequency.
\[GeneratePlanBasedOnFrequency(User, WeeklyFrequency) \rightarrow GeneratedWorkoutPlan \]
\[ \forall (User, WeeklyFrequency): (User \neq null \land WeeklyFrequency \neq null) \Rightarrow (GeneratedWorkoutPlan = true) \]
\textit{Rationale:} Customizing workout plans according to the user's weekly availability fosters a flexible approach to fitness that can accommodate varied lifestyles and commitments, thus promoting regular exercise habits.
\\

UC2: The app shall allow users to design and save their own custom workouts tailored to their preferences.
\[CreateCustomWorkout(User, Preferences) \rightarrow CustomWorkout \]
\[ \forall (User, Preferences):(User \neq null \land Preferences \neq null) \Rightarrow (CustomWorkout = true) \]
\textit{Rationale:} The ability for users to design and save their own workouts caters to individual exercise preferences and goals, providing a personalized fitness experience that can enhance motivation and user satisfaction.
\\

UC3: The system should list all exercises.
\[ \forall  (Exercise:Exercises | Exercise \in ExerciseList) \]
\textit{Rationale:} Providing a comprehensive list of exercises within the system serves as a valuable repository from which users can choose to create varied and targeted workout routines, ensuring a diverse and adaptable exercise regimen.

\subsection{Workout History}
WH1: The app shall display a user’s workout history.
\[DisplayHistory(User) \rightarrow History \]
\[ \forall User: (User \neq null \wedge WorkoutPlan \neq null) \Rightarrow (History \neq null) \]
\textit{Rationale:} Providing users with access to their workout history enables them to track progress over time, gain insights into their fitness journey, and maintain motivation by reflecting on past achievements and areas for improvement.
\\

WH2: The app saves a completed workout.
\[SaveCompletedWorkout(User, CompletedWorkout) \rightarrow SavedWorkout \]
\[ \forall (User, CompletedWorkout): (User \neq null \land CompletedWorkout = true) \Rightarrow (SavedWorkout = true) \]
\textit{Rationale:} The ability to save completed workouts is essential for tracking the user’s progress and adherence to their fitness plan. This feature underpins the app's capability to analyze trends and provide meaningful feedback to the user.

\subsection{Database}
DB1: The app shall have a database to store user account information, workout plans, user feedback, and workout history.
\[StoreData(UserAccount, WorkoutPlan, UserFeedback, WorkoutHistory) \rightarrow StoredData \]
\[ \forall (UserAccount, WorkoutPlan, UserFeedback, WorkoutHistory):\] 
\[(UserAccount \neq null \land WorkoutPlan \neq null \land UserFeedback \neq null \land WorkoutHistory \neq null) \\
\Rightarrow\] \[(StoredData = true) \]
\textit{Rationale:} A central database to store user-related data is fundamental for the app's operation. It ensures that all user interactions, preferences, and history are persistently saved and can be efficiently retrieved, thus providing a seamless and personalized user experience.

\subsection{Additional Features}
AF1: The app should allow users to track their completion of exercises when completing a workout through a live workout session.
\[TrackLiveWorkout(User, PlannedWorkout) \rightarrow LiveTracking \]
\[ \forall (User, PlannedWorkout): (User \neq null \land PlannedWorkout = true) \Rightarrow (LiveTracking = true) \]
\textit{Rationale:} The feature of live workout tracking immerses users in their fitness routine by providing real-time data. This immediate feedback is invaluable for maintaining correct form, pacing, and motivation throughout the exercise session.
\\

AF2: The app allows users to start planned workouts and follow along live.
\[StartLiveWorkout(User, PlannedWorkout) \rightarrow LiveWorkoutSession \]
\[ \forall (User, PlannedWorkout): (User \neq null \land PlannedWorkout = true) \Rightarrow (LiveWorkoutSession = true) \]
\textit{Rationale:} Allowing users to start and follow a planned workout facilitates a structured exercise session. This feature assists in ensuring that workouts are completed as intended, contributing to the user’s consistency and progression in fitness.
\\

AF3: The app allows users to download workouts for offline access.
\[DownloadWorkout(User, SelectedWorkout) \rightarrow OfflineWorkout \]
\[ \forall (User, SelectedWorkout): (User \neq null \land SelectedWorkout \neq null) \Rightarrow (OfflineWorkout = true) \]
\textit{Rationale:} The capability to download workouts for offline use adds flexibility to the app, allowing users to stay on track with their fitness goals regardless of internet connectivity, which is particularly useful for users with an unpredictable or mobile lifestyle.\\

\section{Non-functional Requirements}

\subsection{Look and Feel}

\subsubsection{Appearance Requirements}
\begin{itemize}
\item APR1: The app shall have an intuitive and user-friendly interface with minimalist design.\\
\textit{Rationale:} A minimalist and intuitive interface design is crucial for enhancing user engagement and reducing the learning curve, thus providing a streamlined and focused user experience.
\item APR2: The app shall be compatible with the standard phone aspect ratio, 9:16.\\
\textit{Rationale:} Ensuring the app’s compatibility with different device screens guarantees a consistent and accessible user experience across a wide range of mobile devices.
\item APR3: The app shall provide visual feedback whenever a user action takes place.\\
\textit{Rationale:} Providing visual feedback for user interactions reinforces the app's responsiveness and guides the user through successful command executions and error notifications.
\item APR4: The app shall load data at a time no more than 500 ms so there are no noticeable delays for users.\\
\textit{Rationale:} Fast loading times are imperative to maintain user attention and prevent frustration, thereby contributing to a smoother, more efficient user experience.
\item APR5: The app shall only support portrait mode.\\
\textit{Rationale:} Supporting only portrait mode can simplify the design and development process, ensuring optimal usability in common scenarios where the app is likely to be used, such as one-handed operation during workouts.
\end{itemize}

\subsubsection{Styling Requirements}
\begin{itemize}
\item STR1: The app should have consistent styling and design throughout by using the same colour palette.\\
\textit{Rationale:} Utilizing a consistent colour palette and design elements throughout the app reinforces brand identity and enhances the user's navigational experience by providing a cohesive visual language.
\item STR2: The application’s styling should be modern and similar to iOS apps.\\
\textit{Rationale:} Adopting a modern styling approach, in line with iOS app aesthetics, leverages familiar user interface patterns for iOS users, promoting an intuitive and seamless user experience.
\item STR3: The styling should be visually appealing and align with the app’s logo or theme.\\
\textit{Rationale:} Ensuring that the app's styling complements its logo and theme is crucial for creating a harmonious and attractive visual experience, which can significantly contribute to user engagement and brand recognition.
\end{itemize}

\subsection{Usability and Human Requirements}

\subsubsection{Ease of Use Requirements}
\begin{itemize}
\item EUR1: The system shall be intuitive such that new users can quickly understand basic functions and commands by using intuitive icons.\\
\textit{Rationale:} An intuitive interface with clear icons reduces the learning curve for new users, ensuring that they can quickly become proficient in using the app without frustration, thus enhancing user retention and satisfaction.
\item EUR2: The user interface elements, terminology, and interactions should remain consistent throughout the application’s various pages.\\
\textit{Rationale:} Consistency in UI elements and terminology across the app helps users form reliable expectations about how to interact with the app, reducing confusion and improving overall usability.
\end{itemize}

\subsubsection{Personalization Requirements}
\begin{itemize}
\item PER1: The app will support local date/time formats based on the user’s device settings.\\
\textit{Rationale:} Adapting to the local date and time formats of the user's device settings personalizes the user experience, making it more relevant and easier to understand, thereby enhancing user engagement.
\item PER2: The app shall stay consistent with the personalization settings on the user’s device regarding display and font.\\
\textit{Rationale:} Aligning the app’s display and font settings with the user’s device personalization preferences ensures a seamless experience that respects the user's choices and accessibility needs.
\end{itemize}

\subsubsection{Learning Requirements}
\begin{itemize}
\item LER1: Users shall not require external resources to navigate the application; everything they need to know to use the app should be taught in-app.\\
\textit{Rationale:} Providing users with the necessary tools and information within the app to learn its functionalities ensures a self-sufficient user experience, reducing the need for external support and increasing user autonomy.
\end{itemize}

\subsubsection{Accessibility Requirements}
\begin{itemize}
\item ACR1: The system will follow Apple’s design guidelines regarding accessible applications.\\
\textit{Rationale:} Following Apple’s accessibility guidelines ensures that the application is usable by people with a wide range of physical and cognitive abilities, promoting inclusivity and expanding the app's user base. This commitment to accessibility not only aligns with legal and ethical standards but also enhances user satisfaction and app usability.
\end{itemize}

\subsection{Performance Requirements}
\begin{itemize}
\item PR1: The system shall be capable of handling a minimum of 1000 simultaneous users and processing 200 transactions per minute while maintaining performance standards.\\
\textit{Rationale:} Ensuring the system can support a significant number of simultaneous users and transactions is essential for maintaining a smooth and reliable user experience, preventing system overloads during peak usage times, and fostering user trust in the app’s performance.
\item PR2: The system shall achieve a minimum of 99.9\% availability, ensuring services are accessible to users at all times, excluding scheduled maintenance windows.\\
\textit{Rationale:} A high availability rate minimizes downtime and ensures that users can access the app’s services when needed, which is crucial for user retention and satisfaction, especially for an app designed to be part of daily routines.
\item PR3: The system shall support a data throughput of at least 200 Mbps to ensure data transmission between the server and users is seamless and efficient under peak usage.\\
\textit{Rationale:} Supporting a high data throughput rate is necessary to handle the volume of information exchanged between the server and users efficiently, ensuring that app interactions are fast and seamless, even under heavy user load.
\item PR4: The system shall deliver a response time of no more than 2 seconds for 99\% of the transaction processed under standard operating conditions.\\
\textit{Rationale:} A fast response time is critical for a positive user experience, reducing wait times and making the app feel more responsive and reliable, which is particularly important for retaining users and encouraging frequent use.
\end{itemize}

\subsection{Operational and Environment Requirements}
\begin{itemize}
\item OER1: The system shall be able to be used in a gym/workout facility or home workout room.\\
\textit{Rationale:} Ensuring the app's functionality across various environments, such as gyms, workout facilities, or home workout rooms, maximizes its utility and accessibility, allowing users to maintain their fitness routines in diverse settings without compromise.
\item OER2: The system shall be able to run on an iOS phone.\\
\textit{Rationale:} Designing the system to run on iOS phones leverages the widespread popularity and advanced capabilities of these devices, ensuring a broad user base can access and benefit from the app's features.
\item OER3: The system shall allow the audio output to be transmitted through the device’s speakers or a connected Bluetooth device.\\
\textit{Rationale:} Providing users with the ability to use the device’s speakers or connect to Bluetooth devices for audio output enhances the app's flexibility and usability, catering to personal preferences for audio during workouts.
\item OER4: The system should connect to the internet for back-end services.\\
\textit{Rationale:} Requiring an internet connection for accessing back-end services ensures that the app can offer dynamic, updated content and sync user data across devices, contributing to a seamless and enriched user experience.
\end{itemize}

\subsection{Compliance Requirements}
\begin{itemize}
\item COMR1: The application will adhere to all data protection laws in the region(s) it operates within.\\
\textit{Rationale:} Compliance with local and international data protection laws is crucial for legal operation, building user trust, and safeguarding against penalties or lawsuits, thereby ensuring the app's longevity and credibility.
\item COMR2: The informed consent of users shall be obtained before collecting and processing their personal data.\\
\textit{Rationale:} Securing informed consent before collecting personal data respects user privacy and autonomy, aligning with ethical standards and legal requirements, and fostering a transparent relationship between the app and its users.
\item COMR3: Robust security measures will be put in place to safeguard user data.\\
\textit{Rationale:} Strong security measures protect sensitive user information from unauthorized access, data breaches, and cyber threats, thus ensuring user trust and app integrity.
\item COMR4: The application will be transparent about its data collection.\\
\textit{Rationale:} Being transparent about the nature, purpose, and scope of data collection helps in maintaining an open and trustworthy relationship with users, encouraging their continued engagement with the app.
\item COMR5: The app shall always recommend safe and proven fitness guidance and information.\\
\textit{Rationale:} Recommending only safe and scientifically validated fitness information underpins the app's reliability and effectiveness, contributing to user safety and positive health outcomes.
\item COMR6: The system must state that the application is not responsible for any injuries or accidents resulting from its use.\\
\textit{Rationale:} A clear disclaimer regarding the non-liability of the app for injuries or accidents arising from its use is essential for legal protection and informs users of their responsibility in following the provided fitness guidance safely.
\end{itemize}

\subsection{Security Requirements}
\subsubsection{Access Requirements}

\begin{itemize}
    \item \textbf{ACR1: Strong Password Restrictions}\\
    The app shall enforce strong password restrictions to ensure users are properly protecting their own data.\\
    \textit{Rationale:} Implementing strong password policies is a fundamental aspect of securing user accounts and personal data from unauthorized access, thereby enhancing overall app security and user trust.

    \item \textbf{ACR2: Secure User Authentication}\\
    Users shall authenticate themselves with their credentials securely.\\
    \textit{Rationale:} Secure authentication mechanisms prevent potential security breaches, ensuring that only authorized users can access their accounts and personal information, which is critical for maintaining privacy and security.

    \item \textbf{ACR3: Authorization for Access}\\
    Users will only have access to data and features for which they are authorized.\\
    \textit{Rationale:} Limiting user access to authorized data and features prevents misuse and unauthorized data access, upholding the principles of least privilege and ensuring data integrity and confidentiality.
\end{itemize}


\subsubsection{Integrity Requirements}

\begin{itemize}
    \item \textbf{INR1: Restriction of Sensitive Data}\\
    Sensitive data shall be restricted from the users of the app.\\
    \textit{Rationale:} Limiting access to sensitive data ensures user privacy and trust are maintained, preventing unauthorized access and potential data breaches.
    
    \item \textbf{INR2: Encryption of Data in Transit}\\
    Data between the app and the server should be encrypted when an API call is made.\\
    \textit{Rationale:} Encrypting data during transmission protects it from interception by unauthorized parties, ensuring data integrity and confidentiality.
    
    \item \textbf{INR3: Regular Data Backups}\\
    The system should regularly back up data automatically to prevent data loss in the event of database failures.\\
    \textit{Rationale:} Routine backups safeguard against data loss, ensuring that critical information can be restored and that the app remains operational after unforeseen failures.
    
    \item \textbf{INR4: Local Storage of Unsynced Data}\\
    Unstored data should be stored locally if the data cannot be updated.\\
    \textit{Rationale:} Storing unsynced data locally prevents loss of user progress due to connectivity issues, enhancing user experience by providing resilience and reliability.
    
    \item \textbf{INR5: Backend Validation for Data Uniqueness}\\
    The system must incorporate backend validation to enforce the uniqueness of primary keys associated with user accounts in the database.\\
    \textit{Rationale:} Ensuring the uniqueness of primary keys prevents data corruption and conflicts within the database, maintaining the integrity and reliability of user account information.
    
    \item \textbf{INR6: Progress Preservation Mechanism}\\
    The system must have a mechanism to store the user's progress when completing a live workout, allowing them to resume their workout from the point of interruption when the application is reopened.\\
    \textit{Rationale:} Preserving and allowing the resumption of workout progress enhances user satisfaction by providing a seamless and uninterrupted fitness experience.
    
    \item \textbf{INR7: Clear Warnings About Exercise Safety}\\
    The system should display clear and prominent warnings to users regarding exercise safety.\\
    \textit{Rationale:} Promoting exercise safety through clear warnings helps prevent injuries and reinforces the app’s commitment to user well-being.
\end{itemize}


\subsubsection{Privacy Requirements}

\begin{itemize}
    \item \textbf{PRR1: Encryption of Sensitive Data}\\
    Sensitive data should be encrypted both in transit and at rest to protect it from unauthorized access.\\
    \textit{Rationale:} Encrypting sensitive data at all times ensures the highest level of security, protecting user information from potential cyber threats and unauthorized access, thereby maintaining user trust.
\end{itemize}

\subsubsection{Error Handling Requirements}

\begin{itemize}
    \item \textbf{EHR1: Robust Error Handling}\\
    The system should have robust error handling that doesn't reveal sensitive system information to users in error messages.\\
    \textit{Rationale:} Implementing comprehensive error handling prevents the exposure of sensitive system details that could be exploited by malicious entities, enhancing the overall security posture of the application.
\end{itemize}

\section{Likely Changes}

\begin{itemize}
    \item \textbf{LC1: Adaptation to New iOS Versions}\\
    The application may need to adapt to future versions of iOS to ensure compatibility and leverage new features.\\
    \textit{Rationale:} Staying updated with the latest iOS versions ensures the app remains functional and attractive to users, taking advantage of new technologies and improvements offered by the platform.
    
    \item \textbf{LC2: Integration of New APIs or Services}\\
    The integration of new APIs or third-party services to enhance functionality, such as adding a new payment gateway or integrating a new workout tracking device.\\
    \textit{Rationale:} Expanding the app’s capabilities through additional APIs or services can provide users with new features and better integration with external devices, enriching the user experience.
    
    \item \textbf{LC3: UI/UX Refinements}\\
    Refinements in the UI/UX based on user feedback and usability testing may be required post-launch.\\
    \textit{Rationale:} Continuous improvement of the UI/UX is essential for meeting user expectations and addressing any usability challenges, ensuring the app remains competitive and user-friendly.
    
    \item \textbf{LC4: New Features or Screens}\\
    The addition of new features or screens to accommodate evolving user needs and market trends, such as incorporating a new workout trend or social sharing capabilities.\\
    \textit{Rationale:} Adapting to changing user needs and market trends keeps the app relevant and engaging, encouraging ongoing user interaction and satisfaction.
    
    \item \textbf{LC5: Backend Infrastructure Enhancement}\\
    As the user base grows, enhancing the backend infrastructure to manage increased load, reducing latency, and ensuring smooth user experiences.\\
    \textit{Rationale:} Scalability of the backend infrastructure is vital for maintaining performance and reliability as the number of users increases, ensuring a seamless experience for all.
    
    \item \textbf{LC6: Data Protection and Privacy Adjustments}\\
    Changes in data protection laws or health app regulations may require adjustments in data handling and user privacy features.\\
    \textit{Rationale:} Compliance with legal and regulatory changes is essential for operational legality and protecting user data, reflecting the app’s commitment to privacy and security.
    
    \item \textbf{LC7: Inclusion of New Features or Modifications}\\
    Inclusion of new features or modification of existing ones to stay competitive and meet user demands.\\
    \textit{Rationale:} Continuously evolving the app through new features or modifications ensures it meets current user expectations and market demands, driving growth and user engagement.
\end{itemize}


\section{Unlikely Changes}

\begin{itemize}
    \item \textbf{UL1: Core Functionality and Features}\\
    The basic functionality and core features of the app, such as tracking workouts and providing workout plans, are fundamental and unlikely to change drastically.\\
    \textit{Rationale:} The central premise and utility of the app are defined by its ability to track workouts and generate fitness plans. These features are integral to the app's identity and purpose, making significant changes to them unlikely without altering the app's fundamental value proposition.

    \item \textbf{UL2: Target Audience}\\
    The primary target audience, being fitness enthusiasts and gym-goers, is unlikely to change.\\
    \textit{Rationale:} The app is specifically designed to meet the needs and interests of individuals committed to fitness, which constitutes its core user base. While the app may expand its appeal to a broader audience over time, the core target audience is expected to remain constant due to the app's specialized focus.

    \item \textbf{UL3: Platform Choice (iOS)}\\
    The platform choice (iOS) given that the styling, design, and functionalities are deeply integrated with iOS features.\\
    \textit{Rationale:} The decision to build and optimize the app for iOS is based on leveraging the specific advantages and features of the iOS platform. Unless there's a strategic shift in market focus or platform capabilities, this foundational choice is unlikely to change.

    \item \textbf{UL4: Development Technologies}\\
    The programming languages and core technologies used in development are unlikely to change unless a significant technological shift happens.\\
    \textit{Rationale:} The selection of development languages and technologies is based on current best practices, team expertise, and platform requirements. Significant changes to these choices would likely only occur in response to major advancements in technology or shifts in the development landscape.
\end{itemize}



\section{Traceability Matrices}
Table 1 and Table 2 show the dependencies between the functional requirements. A requirement in the first column that is dependent on a requirement in the first row is marked by an 'X'.


\begin{landscape}

\begin{table}[h!]
\centering
\begin{tabular}{|c|c|c|c|c|c|c|c|c|c|c|c|c|c|c|c|c|c|c|}
 \hline
   & UA1 & UA2 & UA3 & UA4 & UA5 & EB1 & EB2 & EB3 & EB4 & UC1 & UC2 & UC3 & WH1 & WH2 & DB1 & AF1 & AF2 & AF3 \\
 \hline
 UA1 & & & & & & & & & & & & & & & & & & \\
 \hline
 UA2 & X & & & & & & & & & & & & & & & & & \\
 \hline
 UA3 & & X & & & & & & & & & & & & & X & & & \\
 \hline
 UA4 & & & & & & & & & & & & & & & & & & \\
 \hline
 UA5 & X & & & & & & & & & & & & & & & & & \\
 \hline
 EB1 & & & & & & & & & & X & & & & & X & & & \\
 \hline
 EB2 & & & & & & & & & & & X & & & & X & & & \\
 \hline
 EB3 & & & & & & & & & & & & & & & & X & & \\
 \hline
 EB4 & & & & & & & & & & & & & & & & X & & \\
 \hline
 UC1 & & & & & & & & & & & & & & & & & X & \\
 \hline
 UC2 & & & & & & & & & & & & X & & & X & & & \\
 \hline
 UC3 & & & & & & & & & & & & & & & X & & & \\
 \hline
 WH1 & & & & & & & & & & & & & & & X & & & \\
 \hline
 WH2 & & & & & & & & & & & & & X & & X & & & \\
 \hline
 DB1 & & & X & & & & & & & & & & & & & & & \\
 \hline
 AF1 & & & & & & & & X & X & & & & & & & & X & \\
 \hline
 AF2 & & & & & & & & & & X & & & & & & & & \\
 \hline
 AF3 & & & & & & & & & & & & & & & & & & X \\
 \hline
\end{tabular}
\caption{Traceability Matrix Showing the Mappings Between Functional Requirements}
\label{Table:A_trace}
\end{table}

\end{landscape}

}

\begin{table}[h!]
\centering
\begin{tabular}{|c|c|c|c|c|c|}
 \hline
  & UA1 & UA2 & UA3 & UA4 & UA5 \\
 \hline
 UA1 & & & & X &\\
 \hline
 UA2 & X & & & & \\
 \hline
 UA3 & & X & & & \\
 \hline
 UA4 & & & & & \\
 \hline
 UA5 & X & & & & \\
 \hline
 EB1 & & & & & \\
 \hline
 EB2 & & & X & & \\
 \hline
 EB3 & & & & & \\
 \hline
 EB4 & & & & & \\
 \hline
 UC1 & & & & & \\
 \hline
 UC2 & & & & & \\
 \hline
 UC3 & & & & & \\
 \hline
 WH1 & & & & & \\
 \hline
 WH2 & & & & & \\
 \hline
 DB1 & & & X & & \\
 \hline
 AF1 & & & & & \\
 \hline
 AF2 & & & & & \\
 \hline
 AF3 & & & & & \\
\hline
\end{tabular}
\caption{Traceability Matrix Showing the Mappings Between Functional Requirements}
\label{Table:B_trace}
\end{table}



\section{Development Plan}

The development of the fitness application will follow a phased approach to ensure efficient progress and quality assurance. The order of requirements implementation is as follows:

\subsection{Phase 1: Basic Functionality}

In this initial phase, we will focus on implementing the core features required for the app's basic functionality. This includes:

\begin{itemize}
    \item User account management (UA1 - UA5)
    \item Evidence-Based Workout Generation (EB1 - EB4)
    \item User Workout Creation (UC1 - UC3)
\end{itemize}

\subsection{Phase 2: User Experience Enhancement}

During this phase, we will concentrate on improving the user experience by addressing usability and appearance requirements. Key tasks include:

\begin{itemize}
\item Implementing appearance and styling requirements (APR1 - APR5, STR1 - STR3)
\item Enhancing usability and personalization (EUR1 - EUR2, PER1 - PER2)
\item Implementing learning requirements (LER1)
\end{itemize}

\subsection{Phase 3: Performance and Accessibility}

In this phase, we will ensure that the app meets performance, accessibility, and compatibility requirements. This includes:

\begin{itemize}
\item Addressing performance requirements (PR1 - PR4)
\item Ensuring operational and environmental compatibility (OER1 - OER4)
\item Implementing accessibility guidelines (ACR1)
\end{itemize}

\subsection{Phase 4: Security and Compliance}

The final phase will focus on security and compliance with relevant regulations. Key tasks involve:

\begin{itemize}
\item Implementing integrity requirements (INR1 - INR7)
\item Ensuring compliance with data protection laws and user consent (COMR1 - COMR4)
\item Addressing safety and health recommendations (COMR5 - COMR6)
\end{itemize}

\subsection{Phase 5: Likely Changes and Post-launch Refinements}

After the initial development and launch, we will continuously monitor user feedback and market trends. We will be prepared to implement likely changes (LC1 - LC7) and make refinements as needed to enhance the app's features and performance.

\subsection{Phase 6: Unlikely Changes and Long-Term Maintenance}

While the core features (UL1 - UL4) are unlikely to change significantly, we will remain committed to maintaining the app, ensuring its compatibility with future iOS versions, and providing ongoing support and updates. 


\newpage

\bibliographystyle {plainnat}
\bibliography {../../refs/References}



\newpage{}
\section*{Appendix --- Reflection}

The information in this section will be used to evaluate the team members on the
graduate attribute of Lifelong Learning.  Please answer the following questions:

\begin{enumerate}
  \item What knowledge and skills will the team collectively need to acquire to
  successfully complete this capstone project?  Examples of possible knowledge
  to acquire include domain specific knowledge from the domain of your
  application, or software engineering knowledge, mechatronics knowledge or
  computer science knowledge.  Skills may be related to technology, or writing,
  or presentation, or team management, etc.  You should look to identify at
  least one item for each team member.\\

Fitness and Health Domain:\\
Understanding the fitness and health domain, including workout planning, nutrition, and exercise physiology, is essential for designing effective and personalized fitness solutions.

Machine Learning:\\
Our team will need to gain expertise in ML and AI methodologies in order to develop SweatSmart. To train the AI model, we first need to collect quantitative and qualitative data. Using the data, we can build and train the algorithm to generate personalized workouts.

Mobile App Development:\\
Mastery of mobile app development for iOS will be necessary for our team. We plan on using React to develop the front-end and .NET for the back end. All team members will need to learn and/or refine knowledge in these development tools for this project to be successful.

UI/UX Design:\\
Understanding key principles of user interface and user experience design to create intuitive, visually appealing, and user-friendly interfaces will be important. We will use Figma to visualize pages of our app, making it more efficient to code.

Database Management:\\
Skills in database design, data storage, and retrieval are needed to manage user accounts, workout plans, and progress data securely.

  \item For each of the knowledge areas and skills identified in the previous
  question, what are at least two approaches to acquiring the knowledge or
  mastering the skill?  Of the identified approaches, which will each team
  member pursue, and why did they make this choice?


\textbf{Fitness Domain Research:}\\
Team Members: Sam and Jonny

To ensure a comprehensive understanding of fitness practices, we will conduct research by:
\begin{itemize}
    \item Reading at least 5 peer-reviewed articles on the latest fitness practices.
    \item Exploring 10 or more fitness-related blogs each.
    \item Analyzing 3 or more existing fitness applications/websites each.
\end{itemize}

This extensive research will provide us with diverse insights into the current fitness landscape, allowing us to make informed decisions during app development.

\textbf{Consultation with Fitness Experts:}\\
Team Members: Sophie and Daniel

We will consult with fitness experts, including those at McMaster University's kinesiology department, McMaster personal trainers, and potentially faculty members from the Physiotherapy department. This step is crucial for gaining insights and guidance from primary sources who are experts in the field of fitness.

\textbf{Machine Learning Skill Development:}\\
To gain proficiency in machine learning, we will:
\begin{itemize}
    \item Utilize online resources for learning how to create and train AI algorithms.
    \item Seek assistance from professors and industry experts in the field.
\end{itemize}

\textbf{Mobile App Development:}\\

To enhance our mobile app development skills, we will:
\begin{itemize}
    \item Develop a simple, bare-bones app to practice fundamental functions using React and .Net. (Team Members: Sophie Fillion, Sam McDonald). This will be suefil practice to ensure we understand all basic functions and practices for developing a mobile application using React and .Net. 
    \item Practice writing code for mobile platforms (Team Members: Jonny and Daniel). We are already familiar with React and .Net for mobile development, so we will write smaller practice sections and assist Sophie and Sam in developing their basic apps, serving as mentors to ensure a strong understanding of these frameworks.
\end{itemize}


\textbf{UI/UX Design Skill Development:}\\
To improve our UI/UX design skills, we will:
\begin{itemize}
    \item Create design mockups (Team Members: Jonny and Daniel). Justification: Collaborative tutorials and learning from fellow team members are highly effective ways to enhance skills. Jonny, with expertise in UI/UX design, will provide tutorials, allowing the team to align on design principles before commencing app development.
\end{itemize}

\textbf{Database Management Skill Development:}\\
To strengthen our database management skills, we will:
\begin{itemize}
    \item Review class notes on SQL queries from our databases class.
    \item Utilize online resources for additional learning.
\end{itemize}

\end{enumerate}

\end{document}