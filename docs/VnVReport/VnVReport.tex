\documentclass[12pt, titlepage]{article}

\usepackage{booktabs}
\usepackage{tabularx}
\usepackage{hyperref}
\hypersetup{
    colorlinks,
    citecolor=black,
    filecolor=black,
    linkcolor=red,
    urlcolor=blue
}
\usepackage[round]{natbib}
\usepackage{pdflscape}
\usepackage{afterpage}
\usepackage{geometry}

\input{../Comments}
%% Common Parts

\newcommand{\progname}{SFWRENG 4G06} % PUT YOUR PROGRAM NAME HERE
\newcommand{\authname}{Team \#7, Team FAAM, SweatSmart
\\
\\ Daniel Akselrod
\\ Jonathan Avraham
\\ Sophie Fillion
\\ Sam McDonald} % AUTHOR NAMES                  

\usepackage{hyperref}
\hypersetup{colorlinks=true, linkcolor=blue, citecolor=blue, filecolor=blue,
    urlcolor=blue, unicode=false}
\urlstyle{same}
                                


\newcounter{mnum}
\newcommand{\mthemnum}{M\themnum}
\newcommand{\mref}[1]{M\ref{#1}}

\begin{document}

\title{Verification and Validation Report: \progname} 
\author{\authname}
\date{\today}
	
\maketitle

\pagenumbering{roman}

\section{Revision History}

\begin{tabularx}{\textwidth}{p{3cm}p{2cm}X}
\toprule {\bf Date} & {\bf Version} & {\bf Notes}\\
\midrule
Mar. 6, 2024 & 1.0 & first draft of document\\
\bottomrule
\end{tabularx}

~\newpage

\section{Symbols, Abbreviations and Acronyms}

\renewcommand{\arraystretch}{1.2}
\begin{tabular}{l l} 
  \toprule		
  \textbf{symbol} & \textbf{description}\\
  \midrule 
  T & Test\\
  \bottomrule
\end{tabular}\\

\wss{symbols, abbreviations or acronyms -- you can reference the SRS tables if needed}

\newpage

\tableofcontents

\listoftables %if appropriate

\listoffigures %if appropriate

\newpage

\pagenumbering{arabic}

This document ...

\section{Functional Requirements Evaluation}

\section{Nonfunctional Requirements Evaluation}

\subsection{Usability}
		
\subsection{Performance}

\subsection{etc.}
	
\section{Comparison to Existing Implementation}

As the digital fitness landscape overflows with a variety of workout generator and tracker applications, we have plenty of material to compare our workout application. This section aims to dissect how SweatSmart, developed under constraints of time, budget, and expertise, measures against the backdrop of existing, often well-resourced, applications. We do not expect this application to be comparable to some of the fully-funded, regularly updated options out there, but it is still important to make comparisons to learn where we can improve, and where we could improve if we had time. There are so many existing implementations out there, we decided to look mostly at the apps used in the past by the kin academics and professionals we interviewed for our usability testing session on February 28, 2024. 

\subsection{UI Design and Feature Set}

SweatSmart’s UI design is intentionally minimalist, aimed at providing a straightforward user experience. This design philosophy contrasts with more feature-rich apps, where the complexity of options can overwhelm users. Our app includes basic functionalities such as workout logging, a social feature being actively implemented to enhance user interaction, and the addition of guidance videos and muscle group images for a more informative workout experience. These planned enhancements aim to address the gap in instructional content, a limitation in our current offering. All of the above mentioned features appeared in the other implementations that came up in usability testing.

Another feature present in other available applications is the visualization of workout history and progress, something we have noted and are working to implement by the end of the project.

\subsection{Evidence-Based Approach and Limitations}

Central to our development philosophy is an evidence-based approach to fitness, which inherently restricts the scope of personalization we can offer, since the evidence suggests a simple, easy to follow workout plan is much more beneficial than one with lots of features and specifics. This approach has led to a simpler app design, focusing on universally beneficial workouts over highly personalized plans. The broad consensus in fitness evidence supports the benefits of general physical activity, guiding our decision to prioritize straightforward, effective workout options that cater to the average user. When comparing to existing apps, like Peloton or JEfit, our app my appear to lack some personalization aspects. What we have learned, however, is that a lot of these aspects are based in pseudoscience to appear more tailored than they really need. 

\subsection{Planned Enhancements}

Acknowledging feedback received from the Kin experts we interviewed and our comparative analysis, we outline key areas for future development:
\begin{itemize}
    \item Enhancing the social aspect of SweatSmart to engage users through shared experiences and motivational features.
    \item Improving the workout logging UI for a more intuitive and visually appealing process, and ironing out existing bugs that made workout logging more complicated than it needed to be.
    \item Adding guidance videos and images to offer detailed instructions and highlight targeted muscle groups, enhancing the educational value of workouts.
    \item Implementing a system which visualizes important summary data of past workouts and workout history of a user, giving them at-a-glance updates on their progress to keep them motivated.
\end{itemize}

\subsection{Conclusion}

While SweatSmart may not match the depth of features found in more established apps, its development underscores a commitment to evidence-based fitness principles within the constraints we face. Our focus on UI improvements and specific feature enhancements reflects a strategic response to user feedback and a comparative understanding of the market, aiming for meaningful improvements within our app’s defined scope.

This section will not be appropriate for every project.

\section{Unit Testing}

Unit testing in our app is focused on our back-end infrastructure with client-facing endpoints being tested. Every service has at least 1 unit test pertaining to different input combinations an endpoint may receive and their respective responses. Below are all endpoint and helper methods within their respective module.

\subsection{Account Service}
\subsubsection{HashPassword}
\begin{itemize}
     \item CanHashPassword(): Ensures password hashing works
\end{itemize}
\subsubsection{Login}
\begin{itemize}
    \item CanLoginWithEmail(): Ensures login via email works
    \item CanLoginWithUsername(): Ensures login via username works
    \item BadRequestHandled(): Ensures invalid requests are properly handled
    \item IncorrectPasswordHandled(): Ensures wrong password does not login and returns error message
\end{itemize}
\subsubsection{Register}
\begin{itemize}
    \item CanRegisterUser(): Ensures user can register with edge cases for input fields
    \item BadRequestHandled(): Ensured a bad request is handled and returned to user
\end{itemize}
\subsubsection{UpdateUser}
\begin{itemize}
    \item CanUpdateUsername(): Ensures a user can update their username.
    \item CanUpdateEmail(): Ensures a user can update their email.
    \item CanUpdateName(): Ensures a user can update their name.
    \item HandleNonExistentUser(): Ensures update fails if the user does not exist.
\end{itemize}

\subsection{ChatBot Service}
\subsubsection{GetResponse}
\begin{itemize}
    \item CanGetChatResponse(): Tests if the chatbot can provide a response given a valid history.
    \item ReturnsBadRequestOnInvalidInput(): Ensures a bad request is returned for invalid input data.
    \item HandlesAPIFailure(): Ensures service handles failures in the OpenAI API correctly.
    \item MaintainsSystemMessageIntegrity(): Checks if the system message integrity (role and content) is maintained in the response.
    \item FiltersNonFitnessContent(): Ensures the bot responds correctly to non-fitness related queries, as per system message instructions.
\end{itemize}

\subsection{Profile Service}
\subsubsection{UpdateUserPreferences}
\begin{itemize}
    \item CanUpdateUserPreferences(): Ensures user preferences are updated correctly.
    \item HandlesNonExistentUser(): Ensures response for non-existent user is handled correctly.
    \item OverwritesExistingPreferences(): Checks if existing preferences are correctly overwritten.
\end{itemize}
\subsubsection{GetUserPreferences}
\begin{itemize}
    \item CanRetrieveUserPreferences(): Tests if the service can retrieve all user preferences.
    \item HandlesNonExistentUser(): Ensures the service correctly handles a non-existent user.
    \item HandlesNoPreferencesSet(): Ensures the service handles cases where no preferences are set.
\end{itemize}
\subsubsection{GenerateSingularWorkout}
\begin{itemize}
    \item CanGenerateSingularWorkout(): Tests if a single workout is generated correctly.
    \item HandlesNonExistentUser(): Ensures proper handling when the user does not exist.
    \item HandlesNoPreferences(): Ensures correct handling when user preferences are not set.
    \item ReturnsEmptyWorkoutForUnavailableTime(): Checks for proper response when user's time availability is insufficient for any workout.
\end{itemize}
\subsubsection{GenerateWorkoutPlan}
\begin{itemize}
    \item CanGenerateWeeklyWorkoutPlan(): Tests if a weekly workout plan is generated correctly based on frequency.
    \item HandlesNonExistentUser(): Ensures correct response when the user does not exist.
    \item HandlesInvalidFrequency(): Checks handling of frequencies outside the valid range.
\end{itemize}

\subsection{Encryption Helper}
\subsubsection{Initialization}
\begin{itemize}
    \item ConstructorWithValidKey(): Tests if the constructor correctly initializes with a valid encryption key.
    \item ConstructorWithInvalidKey(): Checks the behavior when an invalid or empty key is provided.
\end{itemize}
\subsubsection{Encrypt}
\begin{itemize}
    \item EncryptValidString(): Tests if a valid string is correctly encrypted.
    \item EncryptEmptyString(): Ensures that an empty string is handled appropriately.
    \item EncryptNullString(): Checks the response to a null string input.
\end{itemize}
\subsubsection{Decrypt}
\begin{itemize}
    \item DecryptValidString(): Tests if a valid encrypted string is correctly decrypted.
    \item DecryptEmptyString(): Ensures that an empty string is handled appropriately.
    \item DecryptNullString(): Checks the response to a null string input.
    \item DecryptCorruptedData(): Tests the behavior when decrypting corrupted or altered data.
\end{itemize}
\subsubsection{Integration}
\begin{itemize}
    \item EncryptDecryptCycle(): Validates that a string can be encrypted and then decrypted back to its original form.
\end{itemize}

\subsection{Social Service}
\subsubsection{GetUserProfile}
\begin{itemize}
    \item GetUserProfileWithValidId(): Tests if a user profile is correctly retrieved with a valid user ID.
    \item GetUserProfileWithInvalidId(): Checks the behavior when an invalid or non-existing user ID is provided.
\end{itemize}
\subsubsection{FilterUsers}
\begin{itemize}
    \item FilterUsersWithValidCriteria(): Tests if users are correctly filtered based on provided criteria.
    \item FilterUsersWithEmptyCriteria(): Ensures that appropriate results are returned when an empty filter is applied.
    \item FilterUsersWithNullCriteria(): Checks how the method behaves when null is passed as filter criteria.
    \item PrioritizeFriendsInResults(): Validates if friends are prioritized in the filtered results.
    \item LimitNumberOfResults(): Ensures that the number of results returned is within the specified limit.
\end{itemize}

\section{Changes Due to Testing}

\wss{This section should highlight how feedback from the users and from 
the supervisor (when one exists) shaped the final product.  In particular 
the feedback from the Rev 0 demo to the supervisor (or to potential users) 
should be highlighted.}

\section{Automated Testing}
		
\section{Trace to Requirements}
The traceability between the functional requirements and the test cases is shown in Table 1. The traceability between the nonfunctional requirements and the test cases is shown in Table 2, 3, and 4. 

\afterpage{
\newgeometry{left=1cm, right=0cm, top=0cm, bottom=0cm}
\begin{landscape}

\begin{table}[h!]
\footnotesize
\centering
\begin{tabular}{|c|c|c|c|c|c|c|c|c|c|c|c|c|c|c|c|c|c|c|c|c|c|c|}
 \hline
   & UA1 & UA2 & UA3 & UA4 & UA5 & AI1 & AI2 & AI3 & AI4 & AI5 & UC1 & UC2 & UC3 & UC4 & WH1 & WH2 & DB1 & AF1 & AF2 & AF3 & AF4 & AF5\\
 \hline
 STF-UA-1 & X & & & X & X & & & & & & & & & & & & & & & & & & 
 \hline
 STF-UA-2 & X & & & & & & & & & & & & & & & & & & & & & & 
 \hline
 STF-UA-3 & X & & & & & & & & & & & & & & & & & & & & & & 
 \hline
 STF-UA-4 & & & & X & X & & & & & & & & & & & & & & & & & & 
 \hline
 STF-UA-5 & & X & & & & & & & & & & & & & & & & & & & & & 
 \hline
 STF-UA-6 & & X & & & & & & & & & & & & & & & & & & & & & 
 \hline
 STF-UA-7 & & X & & & & & & & & & & & & & & & & & & & & & 
 \hline
 STF-UA-8 & & X & & & & & & & & & & & & & & & & & & & & & 
 \hline
 STF-UA-9 & & & X & & & & & & & & & & & & & & & & & & & & 
 \hline
 STF-AI-1 & & & & & & X & & & & & & & & & & & & & & & & & 
 \hline
 STF-AI-2 & & & & & & & X & & & & & & & & & & & & & & & & 
 \hline
 STF-AI-3 & & & & & & & & X & & & & & & & & & & & & & & & 
 \hline
 STF-AI-4 & & & & & & & & & X & & & & & & & & & & & & & & 
 \hline
 STF-AI-5 & & & & & & & & & & X & & & & & & & & & & & & & 
 \hline
 STF-UC-1 & & & & & & & & & & & X & & & & & & & & & & & & 
 \hline
 STF-UC-2 & & & & & & & & & & & & X & & & & & & & & & & & 
 \hline
 STF-UC-3 & & & & & & & & & & & & & X & & & & & & & & & & 
 \hline
 STF-UC-4 & & & & & & & & & & & & & X & & & & & & & & & & 
 \hline
 STF-UC-5 & & & & & & & & & & & & & & X & & & & & & & & & 
 \hline
 STF-WH-1 & & & & & & & & & & & & & & & X & & & & & & & & 
 \hline
 STF-WH-2 & & & & & & & & & & & & & & & & X & & & & & & & 
 \hline
 STF-DB-1 & & & & & & & & & & & & & & & & & X & & & & & & 
 \hline
 STF-AF-1 & & & & & & & & & & & & & & & & & & X & & X & & & 
 \hline
 STF-AF-2 & & & & & & & & & & & & & & & & & & & X & & & & 
 \hline
 STF-AF-3 & & & & & & & & & & & & & & & & & & & & & X & & 
 \hline
 STF-AF-4 & & & & & & & & & & & & & & & & & & & & & & X &
\hline
\end{tabular}
\caption{Traceability Matrix Between Test Cases and Functional Requirements}
\label{Table:A_trace}
\end{table}

\begin{table}[h!]
\footnotesize
\centering
\begin{tabular}{|c|c|c|c|c|c|c|c|c|c|c|c|c|c|c|c|c|}
 \hline
   & APR1 & APR2 & APR3 & APR4 & APR5 & STR1 & STR2 & STR3 & EUR1 & EUR2 & PER1 & PER2 & LER1 & UPR1 & UPR2 & ACR1 \\
 \hline
 STN-APR-1 & X & & & & & & & & & & & & & & & \\
 \hline
 STN-APR-2 & & X & & & & & & & & & & & & & & \\
 \hline
 STN-APR-3 & & & X & & & & & & & & & & & & & \\
 \hline
 STN-APR-4 & & & & X & & & & & & & & & & & & \\
 \hline
 STN-APR-5 & & & & & X & & & & & & & & & & & \\
 \hline
 STN-STR-1 & & & & & & X & & & & & & & & & & \\
 \hline
 STN-STR-2 & & & & & & & X & & & & & & & & & \\
 \hline
 STN-STR-3 & & & & & & & & X & & & & & & & & \\
 \hline
 STN-EUR-1 & & & & & & & & & X & & & & & & & \\
 \hline
 STN-EUR-2 & & & & & & & & & & X & & & & & & \\
 \hline
 STN-PER-1 & & & & & & & & & & & X & & & & & \\
 \hline
 STN-PER-2 & & & & & & & & & & & & X & & & & \\
 \hline
 STN-LER-1 & & & & & & & & & & & & & X & & & \\
 \hline
 STN-UPR-1 & & & & & & & & & & & & & & X & & \\
 \hline
 STN-UPR-2 & & & & & & & & & & & & & & & X & \\
 \hline
 STN-ACR-1 & & & & & & & & & & & & & & & & X \\
\hline
\end{tabular}
\caption{Traceability Matrix Between Test Cases and Nonfunctional Requirements (Part 1)}
\label{Table:A_trace}
\end{table}

\begin{table}[h!]
\footnotesize
\centering
\begin{tabular}{|c|c|c|c|c|c|c|c|c|c|c|c|c|c|c|c|c|c|}
 \hline
   & PR1 & PR2 & PR3 & PR4 & OER1 & OER2 & OER3 & OER4 & MSR1 & MSR2 & MSR3 & COMR1 & COMR2 & COMR3 & COMR4 & COMR5 & COMR6\\
 \hline
 STN-PR-1 & X & & & & & & & & & & & & & & & & \\
 \hline
 STN-PR-2 & & X & & & & & & & & & & & & & & & \\
 \hline
 STN-PR-3 & & & X & & & & & & & & & & & & & & \\
 \hline
 STN-PR-4 & & & & X & & & & & & & & & & & & & \\
 \hline
 STN-OER-1 & & & & & X & & & & & & & & & & & & \\
 \hline
 STN-OER-2 & & & & & & X & & & & & & & & & & & \\
 \hline
 STN-OER-3 & & & & & & & X & & & & & & & & & & \\
 \hline
 STN-OER-4 & & & & & & & & X & & & & & & & & & \\
 \hline
 STN-MSR-1 & & & & & & & & & X & & & & & & & & \\
 \hline
 STN-MSR-2 & & & & & & & & & & X & & & & & & & \\
 \hline
 STN-MSR-3 & & & & & & & & & & & X & & & & & & \\
 \hline
 STN-COMR-1 & & & & & & & & & & & & X & & & & & \\
 \hline
 STN-COMR-2 & & & & & & & & & & & & & X & & & & \\
 \hline
 STN-COMR-3 & & & & & & & & & & & & & & X & & & \\
 \hline
 STN-COMR-4 & & & & & & & & & & & & & & & X & & \\
 \hline
 STN-COMR-5 & & & & & & & & & & & & & & & & X & \\
 \hline
 STN-COMR-6 & & & & & & & & & & & & & & & & & X \\
\hline
\end{tabular}
\caption{Traceability Matrix Between Test Cases and Nonfunctional Requirements (Part 2)}
\label{Table:A_trace}
\end{table}


\begin{table}[h!]
\footnotesize
\centering
\begin{tabular}{|c|c|c|c|c|c|c|c|c|c|c|c|c|c|c|}
 \hline
   & ACR1 & ACR2 & ACR3 & INR1 & INR2 & INR3 & INR4 & INR5 & INR6 & INR7 & INR8 & PRR1 & EHR1 & EHR2 \\
 \hline
 STN-ACR-1 & X & & & & & & & & & & & & & \\
 \hline
 STN-ACR-2 & & X & & & & & & & & & & & & \\
 \hline
 STN-ACR-3 & & & X & & & & & & & & & & & \\
 \hline
 STN-INR-1 & & & & X & & & & & & & & & & \\
 \hline
 STN-INR-2 & & & & & X & & & & & & & & & \\
 \hline
 STN-INR-3 & & & & & & X & & & & & & & & \\
 \hline
 STN-INR-4 & & & & & & & X & & & & & & & \\
 \hline
 STN-INR-5 & & & & & & & & X & & & & & & \\
 \hline
 STN-INR-6 & & & & & & & & & X & & & & & \\
 \hline
 STN-INR-7 & & & & & & & & & & X & & & & \\
 \hline
 STN-INR-8 & & & & & & & & & & & X & & & \\
 \hline
 STN-PRR-1 & & & & & & & & & & & & X & & \\
 \hline
 STN-EHR-1 & & & & & & & & & & & & & X & \\
 \hline
 STN-EHR-2 & & & & & & & & & & & & & & X \\
\hline
\end{tabular}
\caption{Traceability Matrix Between Test Cases and Nonfunctional Requirements (Part 3)}
\label{Table:A_trace}
\end{table}

\end{landscape}
}
\restoregeometry

		
\section{Trace to Modules}	
Table 5 shows the traceability between the modules and the requirements.

\begin{description}
\item [\refstepcounter{mnum} \mthemnum \label{mUP}:] User Profile Management Module
\item [\refstepcounter{mnum} \mthemnum \label{mWG}:] Workout Generation Module
\item [\refstepcounter{mnum} \mthemnum \label{mDM}:] Data Management Module
\item [\refstepcounter{mnum} \mthemnum \label{mUI}:] User Interaction Module
\item [\refstepcounter{mnum} \mthemnum \label{mRF}:] Reporting and Feedback Module
\item [\refstepcounter{mnum} \mthemnum \label{mCI}:] ChatBot Integration Module
\item [\refstepcounter{mnum} \mthemnum \label{mAS}:] Algorithm Selection Module
\item [\refstepcounter{mnum} \mthemnum \label{mIF}:] Input Format Module
\end{description}

\begin{table}[H]
\centering
\begin{tabular}{p{0.2\textwidth} p{0.6\textwidth}}
\toprule
\textbf{Requirement} & \textbf{Modules}\\
\midrule
UA1 & \mref{mUP}, \mref{mDM}, \mref{mUI}, \mref{mIF}\\
UA2 & \mref{mUP}, \mref{mDM}, \mref{mUI}, \mref{mAS}\\
UA3 & \mref{mUP}, \mref{mDM}, \mref{mUI}, \mref{mAS}\\
UA4 & \mref{mUP}, \mref{mUI}, \mref{mAS}\\
UA5 & \mref{mUP}, \mref{mUI}, \mref{mAS}\\
AI1 & \mref{mUP}, \mref{mWG}, \mref{mDM}, \mref{mUI}, \mref{mAS}\\
AI2 & \mref{mWG}, \mref{mDM}, \mref{mAS}\\
AI3 & \mref{mWG}, \mref{mDM}, \mref{mRF}, \mref{mAS}\\
AI4 & \mref{mDM}, \mref{mUI}, \mref{mCI}\\
AI5 & \mref{mDM}, \mref{mUI}, \mref{mRF}, \mref{mAS}\\
UC1 & \mref{mUP}, \mref{mWG}, \mref{mDM}, \mref{mUI}\\
UC2 & \mref{mUI}, \mref{mRF}\\
UC3 & \mref{mUP}, \mref{mDM}, \mref{mUI}\\
UC4 & \mref{mDM}\\
WH1 & \mref{mUP}, \mref{mDM}\\
WH2 & \mref{mDM}, \mref{mRF}\\
DB1 & \mref{mUP}, \mref{mWG}, \mref{mDM}, \mref{mUI}, \mref{mRF}\\
AF1 & \mref{mUI}, \mref{mAS}\\
AF2 & \mref{mDM}, \mref{mUI}\\
AF3 & \mref{mUP}, \mref{mWG}, \mref{mDM}, \mref{mUI} \\
AF4 & \mref{mUP}, \mref{mDM} \\
AF5 & \mref{mUP}, \mref{mDM}, \mref{mUI}\\
\bottomrule
\end{tabular}
\caption{Trace Between Requirements and Modules}
\label{TblRT}
\end{table}

\section{Code Coverage Metrics}

\bibliographystyle{plainnat}
\bibliography{../../refs/References}

\newpage{}
\section*{Appendix --- Reflection}

The information in this section will be used to evaluate the team members on the
graduate attribute of Reflection.  Please answer the following question:

\begin{enumerate}
  \item In what ways was the Verification and Validation (VnV) Plan different
  from the activities that were actually conducted for VnV?  If there were
  differences, what changes required the modification in the plan?  Why did
  these changes occur?  Would you be able to anticipate these changes in future
  projects?  If there weren't any differences, how was your team able to clearly
  predict a feasible amount of effort and the right tasks needed to build the
  evidence that demonstrates the required quality?  (It is expected that most
  teams will have had to deviate from their original VnV Plan.)
\end{enumerate}

\end{document}
